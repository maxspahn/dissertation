\chapter{Related Works} % (fold)
\label{cha:state}

\blfootnote{
  \partscontentfootnote
  \begin{itemize}
    \item \trodynamic
    \item \icraautotuning
    \item \icracoupled
  \end{itemize}
}

% There is a split between different embodiments of robotic systems.
% But, there is also a split between global and local
% planning methods. The global planning methods are usually


\begin{abstract}
In the previous chapter, we have defined \ac{tg} as the problem of finding a
sequence of actions that will move the robot from its current state to a
desired state. As this thesis focuses on \ac{tg} for mobile manipulation,
relevant literature must therefore include works from different fields, ranging
from autonomous driving and mobile robotics to manipulation. Besides, \ac{tg} in
dynamic environments generally includes path planning and local \ac{tg}.
Therefore, this chapter summarizes approaches ordered on a scale of reactivity,
that is the frequency at which trajectories are computed. Starting with
controller-like \ac{tg} methods, we then move in order of increasing
time-horizon up until global path planning.
\end{abstract}


\newpage

% The problem of \ac{tg} is paramount to robotic systems,
% i.e., its solution is required in the context of mobile robots,
% drones, manipulators, and consequently mobile manipulation
% systems. Historically, proposed solutions to \ac{tg} were
% developed in parallel for the individual embodiments of
% robotic systems.

\Acl{tg} methods aim at finding connections between two
robot states
that are collision-free at the
moment of planning and potentially optimal.
In the context of changing environments, 
it is useful to place different methods
on a scale of reactivity, usually measured by the achieved
compute frequency, see \cref{fig:reactivity_scale}.
At the most reactive end, we find low
level controllers and on the least reactive end, we find
global path planning methods. In most robotics planning
pipelines, a global path planner is guiding the local
\ac{tg} method.
This thesis focuses on the reactive end of the scale, and
therefore we will only give a brief overview of global path
planning methods. 
%
\begin{figure}[h]
  \centering
  \input{src/state/img/reactivity_scale.pdf_tex}
  \caption{Reactivity scale of different \ac{tg} methods
  revised in this thesis. Global methods, such as \ac{rrt}
  \cite{Karaman2011} are at the least reactive end. More
  reactive methods, such as \ac{mpc}
  \cite{hewing2020learning}, \ac{apf} \cite{Khatib1985} and
  low-level control methods such as \ac{ic} \cite{hogan1985impedance}
  are at the most reactive end.}
  \label{fig:reactivity_scale}
\end{figure}

\section{Impedance control}%
\label{sec:impedance_control}

Impedance control is a widely used control method in
robotics \cite{hogan1985impedance,abu2020variable}. In impedance
control, the controlled system is modeled as a
mass-spring-damper system. The specific case of Cartesian
impedance control was proposed in \cite{albu2002cartesian} and
could be characterized as a local \ac{tg} method, because of
its ability cope with various goal specifications and its
safety properties in the proximity of humans
~\cite{van2022disagreement,hjorth2024enabling}.
Additionally, impedance control can be used in collaborative
settings, e.g., when lifting large objects together with
humans \cite{abu2020variable}. More recent works on
impedance control use variable impedance parameters to adapt
the robot's behavior to the task at hand
\cite{abu2020variable}. Often, the impedance controller is
modified by human feedback to improve perceived safety and
increase reliability of the task execution with human
feedback \cite{lachner2022shaping,franzese2021ilosa}.

\section{Reactive trajectory generation}%
\label{sec:reactive_trajectory_generation}

In this dissertation, we refer to \textit{reactive \ac{tg}}
as methods that rely on local information about the
environment and the robot that compute actions without a
time horizon into the future. The most prominent example of
reactive \ac{tg} is the \acf{apf} method as we will
see by visiting the different embodiments of robotic
systems.

\paragraph{Robotic arms}

Pioneer works on \ac{tg} for robotic arms
employed \ac{apf} for collision avoidance
\cite{Khatib1985,khatib1986real,park2008movement}. Building
on the previous, \cite{Haddadin2011} introduced the Circular
Field method to address dynamic collision avoidance. To
ensure collision avoidance for the end-effector when
grasping a moving obstacle, \cite{Du2018} employed a
repulsive vector.
Velocity scaling of trajectories in the presence of 
contact forces was addressed in \cite{haddadin2010real}.
Reactive \ac{tg} can also be formulated as an optimization
problem. In this case, the optimization problem is solved
at each time step. While statements about global optimality
are not possible, different objectives can be encoded in the
cost function and avoidance can be integrated using
constraints. By maximizing manipulability, the robot remains
flexible to changes in the environment while reaching a goal
pose \cite{haviland2020purely}. The work was later extended
to integrate moving and static obstacles
\cite{haviland2021neo}.

\paragraph{Mobile robots}

The method of \ac{apf} was also used for \ac{tg} of mobile
robotics after its introduction for robotic arms.
For non-holonomic mobile robots, it was shown that \ac{apf}
generally exhibits multiple equilibrium points which are not
necessarily stable \cite{urakubo2018stability}.
Different to \ac{apf}, the dynamic window approach
\cite{Fox1997} and its new variant proposed in
\cite{Zhang2019} have proven to be efficient in generating
smooth trajectories for mobile robots in static and dynamic
environments. To navigate among pedestrians,
\cite{Ferrer2013} introduced the Social Forces model
imitating the human navigation behavior and using it as
navigation policy for the robot.  Yet, Social Forces and its
variants rely on handcrafted functions limiting their
ability to handle more complex navigation scenarios. To deal
with numerous agents, \textit{ORCA} was proposed in
\cite{VanDenBerg2011} and later extended for non-holonomic
bases in \cite{Alonso-Mora2012a}.


While, these reactive \ac{tg} methods capture more
information that pure controllers, they often rely on 
scalar objectives functions to weigh different components.

\subsection{Geometries of trajectories}
\label{sec:geometries_of_trajectories}

Reactive \ac{tg} approaches, such as \ac{apf} theory,
may lead to contradicting behaviors, as individual policies
are \textit{fighting} against each other \cite{Ratliff2018}.
To overcome this shortcoming, some geometric approaches make
an explicit distinction between the importance metric and
the policy of individual behaviors. The metric can often be
interpreted as a measure of curvature on the optimization
manifold. The tuple of metric and policy defines
for each \textit{behavior} a dynamical system.
Formulating \ac{tg} as a dynamical system is a widely used
approach
\cite{khansari2012dynamical,huber2023avoidance}, for which
\ac{osc} \cite{Khatib1987a} and \ac{dmp}
\cite{ijspeert2013dynamical} are special cases.
%
%Based on the findings of contracting metrics for non-linear
%control design \cite{l2,l3}, geometric approaches design the motion
%generation such that convergence is inherent to the problem formulation
%rather than imposing them on the solution process. Practically, individual
%constraints to the motion planning problem shape the optimization manifold
%so that the solution is accessible through the solution of simple differential
%equation. An example for shaping the optimization
%manifold is seen in \cref{fig:spec_combination}.

%
\begin{figure}[h]
  \centering
  \begin{subfigure}{0.33\linewidth}
    \centering
    \includesvg[width=0.9\textwidth]{superposition_0}
    \caption{}
    \label{subfig:trajectory_obst1}
  \end{subfigure}%
  \begin{subfigure}{0.33\linewidth}
    \centering
    \includesvg[width=0.9\textwidth]{superposition_1}
    \caption{}
    \label{subfig:trajectory_obst2}
  \end{subfigure}%
  \begin{subfigure}{0.33\linewidth}
    \centering
    \includesvg[width=0.9\textwidth]{superposition_2}
    \caption{}
    \label{subfig:trajectory_both_obstacles}
  \end{subfigure}
  \caption{
    Combining different avoidance behaviors using optimization fabrics. The
    components defining collision avoidance with single obstacles (a,b) are
    combined in (c). Obstacles are shown in black. Trajectories of the point
    robot are shown in blue.
  }
  \label{fig:spec_combination}
\end{figure}
%
A more general framework was introduced as \acf{rmp} in
\cite{Ratliff2018,Cheng2018}. \Ac{rmp} represent a
natural way of combining multiple policies into one joined policy.
\ac{rmp} define individual sub-tasks of the motion planning as
differential equations (\textit{spectral semi sprays} or
\textit{specs} for short) of second order and propose the
\textit{pullback} and \textit{summation} operators to combine multiple policies
in the configuration space. As subtasks can be defined in arbitrary manifolds
of the configuration space, \ac{rmp} generalize operational space
control~\cite{Khatib1987}. The resulting behavior of
\ac{rmp} was reported to be
intuitive while keeping computational costs low~\cite{Ratliff2018}. The concept
of \ac{rmp} was used in~\cite{Cheng2018,Cheng2020} to form RMP-Flow, a motion
planning algorithm that is shown to be conditionally stable and invariant
across robots. An \ac{rmp} adaptation was proposed for non-holonomic robots
in~\cite{Meng2019}. By incorporating the kinematic constraint into the root
equation of the \ac{rmp}, the computed policy is applicable to non-holonomic robots.
Besides, that work proposed a neural net to learn the collision avoidance task
components. 

Although \ac{rmp} have proven to be a powerful tool for
motion generation, it was reported to require intuition and
experience during tuning~\cite{Ratliff2020}. \Acf{fabrics}
with Finsler structures as metric generators simplify the
motion design as the conditions for stability and
convergence are inherent to the definition of Finsler
structures~\cite{Ratliff2020,Ratliff2021}. Opposed to
\ac{rmp}, where the metric is typically user-defined,
fabrics derive Finsler metrics from artificial energies,
similar to approaches from control design, \cite{l2,l3},
using the Euler-Lagrange-Equation from geometric mechanics.

The theory of \ac{fabrics} was tested on several simple
kinematic chains in~\cite{Ratliff2020,Ratliff2021}. As
\ac{fabrics} design motion as a summation of several
differential equations, each representing a specific
constraint to the motion, it is possible to sequentially
design motion~\cite{Ratliff2020}. This procedure allows to
carefully tune individual components without harming the
others. The application to a planar arm in a goal-reaching
setup was successfully tested in~\cite{Ratliff2020}. Here,
the authors illustrated how the resulting motion can be
modified arbitrarily by the user by adding additional
constraints or preferences. Although \ac{fabrics} generalize
the concept of \ac{rmp} and make it accessible to a broader
audience by decreasing the intuition and expertise required,
they have not yet been applied to a wide range of robots. 

The reason for this lack of application of \ac{fabrics} is
twofold. First, all the above-mentioned methods are reactive
and highly local methods, thus making them prone to local
minima \cite{bhardwaj2022storm}. As \ac{rmp} and
\ac{fabrics} do not incorporate path following, integration
of global path planning to overcome local minima is not
possible to this date. Second, \ac{fabrics} and \ac{rmp} do
not make use of velocity estimates of obstacles but rely
purely on their high reactivity in dynamic environments. As
for other trajectory optimization techniques, motion
estimates could benefit \ac{fabrics} (and \ac{rmp}) to
result in even smoother motion and allow applications in
such environments. 


\section{Receding-horizon trajectory optimization}%
\label{sec:receding_horizon_trajectory_optimization}

Methods formulating local \ac{tg} as an optimization problem with a finite
discrete time horizon are known under the name of receding-horizon trajectory
optimization. In line with most literature in robotics, we will refer to such
methods as \ac{mpc}.
Generally, several objectives are encoded in the scalar
objective function, dynamics
are formulated as equality constraints and inequality constraints ensure
collision avoidance and joint limit avoidance. The dynamics for this problem can
include the full dynamics model or simple integrating
schemes \cite{hewing2020learning}.
By explicitly solving the constrained optimization problem, this approach yields
formal guarantees on stability. Stability for \ac{mpc} is proven by formulating
an appropriate Lyapunov function and showing that the finite time-horizon
formulation with an appropriate terminal cost results in the same stability as
the corresponding infinite time-horizon formulation
\cite{l1,l4,keerthi1988optimal}. 

Despite these results, formal stability
guarantees in such environments are challenging, as appropriate terminal cost
functions are often not commutable or too conservative. Besides, the
computational costs scale with the degrees of freedom, restricting real-time
applicability to simple dynamics and environment models \cite{Spahn2021}.

Some \ac{mpc} formulations are non-linear and can be analyzed
using methods from non-linear control. When analyzing non-linear control
system, Riemannian energies lead to more detailed stability results than
Lyapunov functions. By investigating the variation around the generated
trajectory and its contracting towards the desired trajectory, some control
designs show exponential stabilizing properties \cite{l2}. These
findings have been applied to tracking control problems \cite{l3}.

\paragraph{Mobile robots}

Early works on \ac{mpc} for mobile robots focus on goal
reaching in static environments \cite{howard2010receding}.
Acknowledging the importance of global path following, most
\ac{mpc} formulations rely on contour errors in the
objective function to ensure minimal tracking errors
\cite{lam2010model,brown2017safe}.

MPC schemes were proposed for mobile robots and autonomous vehicles in
\cite{Brito2019} and \cite{Schwarting2018} allowing to optimize over a
prediction horizon and avoid, in advance, dynamic obstacles.  Several 3D MPC
formulation were proposed for drones to enable safe motion through cluttered
environments \cite{Tordesillas2019,Liu2017}.

\paragraph{Robotic arms}

Early implementations of \ac{mpc} for industrial robotics
arms were presented in \cite{faulwasser2016implementation}.
Especially in the context of robotic arms, uncertainty in
the robot model might harm the performance. To overcome this
limitation, Gaussian Processes were proposed in
\cite{hewing2019cautious,carron2019data} to offline learn
the mismatch between model and real system. The learned
model was then used to adapt the \ac{mpc} scheme during
runtime \cite{carron2019data}.

\paragraph{Mobile Manipulators}

In the context of mobile manipulation, less research focused on collision
avoidance in dynamic environments, including changing scenes and moving
obstacles. A real-time controller using MPC was presented in \cite{Ide2011}, in
which either a holonomic or a non-holonomic base was combined with a
two-degree-of-freedom robotic arm mounted onto the base. Although hardware
constraints were respected, no collision avoidance was considered.  An MPC
formulation for mobile manipulators with holonomic bases that allows collision
avoidance was presented in \cite{Avanzini2015}. The perceived obstacles were
translated into a set of spheres to be respected by the MPC scheme. The proposed
approach used dynamically changing weights to change between arm motion and
locomotion, resulting in a locked arm during navigation. A different weight
setting was used to perform motion underneath a horizontal bar with an a priori
position. The work is extended to non-holonomic bases and includes object
detection in moving underneath the horizontal bar \cite{Avanzini2018}. 


\paragraph{Conclusion}

Receding-horizon trajectory optimization methods are a
powerful tool for local \ac{tg} in dynamic environments.
Some realizations provide formal stability guarantees and can be analyzed
using methods from non-linear control. Therefore, \ac{mpc}
approaches are widely used in the context of autonomous
driving. For manipulators, model inaccuracies degrade
performance and learning methods are often used to
compensate for this. However, the computational costs scale
with the degrees of freedom and restrict real-time
applicability to simple dynamics and environment models.

\section{Path planning}
\label{sec:path_planning}

Past works devoted to path planning can be divided into two
main categories: optimization-based and sampling-based
methods \cite{LaValle2006,Mukadam2017}. Sampling-based path
planners generate random configurations until a valid path
between an initial configuration and a set of goal
configurations is found \cite{Karaman2011}. Sampling-based
methods, such as \acf{rrt}
\cite{Webb2013,Kleinbort2019,Kuffner2000} and \acf{prm}
\cite{Hsu2002,Faust2017} are highly
efficient at generating paths for systems with high-degree
of freedom.

\paragraph{Robotic arms}%

Motion planning problems are usually defined by goals in
arbitrary task spaces, such as the 3D Euclidean space or
end-effector poses. For mobile robots, where task space and
configuration space are often identical \cite{LaValle2006},
the mapping from task space to configuration space is
straightforward. However, in the context of manipulation,
tasks can be regarded as constraints to the motion planning
problem. Conventional approaches to motion planning rely on
inverse kinematics to transform task constraints into sets
of configurations. There is abundant research on
solving the resulting path planning problem in the
configuration space \cite{LaValle2006,Rickert2014}. This
thesis focuses on methods to solve the \ac{tg} in arbitrary
task spaces. Therefore, this section is limited to
sampling-based methods that directly integrate tasks into
the sampling process, rather than relying on the inverse
kinematics.

Several methods have been proposed to
directly integrate task constraints into the sampling phase.
\cite{Stilman2010} proposed a method to iteratively push a
random sample to the manifold adhering to the task
constraint. The notion of task constraints was later
extended to task space regions to define soft constraints
for individual task components~\cite{Berenson2011}.
\cite{Kingston2019} proposed scalar-valued functions to
represent task constraints for sampling-based planning. As
all the above-mentioned methods rely on implicitly
constrained sampling in the joint space, they exhibit high
computational time, which is especially harmful to
real-world applications \cite{qureshi2021constrained}.

\paragraph{Mobile Manipulators}

Despite abundant research in trajectory planning for mobile robots and robotic
arms, few works focused on coupling both systems' control. It was shown that
coupling the base and the robotic arm motion leads to a considerable reduction
of total operational time and smoother motions \cite{Thakar2018, Thakar2019}.
Nevertheless, these methods were designed for static environments and did not
allow real-time collision avoidance. Furthermore, trajectory planning for the
coupled system is a precondition for effective interactive navigation, including
opening doors \cite{Jain2009, Chitta2010} or moving obstacles out of the way
\cite{Li2019}.


\section{Environment representations}
\label{sec:environment_representations}

All above described methods rely on some form of environment
representation. For example, the inequality constraints in
\ac{mpc} formulations responsible for collision avoidance
rely on a distance function between the robot and the
obstacles. Path planning methods, as described above, 
also require an environment representation, such that
collision checking can accept or reject new samples. In
geometric approaches, such as \ac{apf}, \ac{rmp} and
\ac{fabrics}, distance functions are also necessary.
For primitive shapes, such as spheres and boxes, distances
can be computed in closed form. When working in dynamic
environments, robots perceive their environment using vision
sensors, such as cameras, or range sensors, such as LiDAR.
Classifying the environment into primitive shapes is a
challenging task, especially when the environment is
cluttered and parts are occluded. A different approach to
collision avoidance relies on more implicit representations
of the environments --the most implicit would be raw sensor
data. In the following, we recall some works that focus on
such representations in the context of \ac{tg}.

\subsection{Implicit representations}
\label{sub:implicit_representations}

For trajectory planning using short-term optimization,
unoccupied space constraints limit robot movement, proven
useful for mobile robots in cluttered areas
\cite{Brito2019}. In drone flight, a similar concept
generates safe flight zones along a global path
\cite{Liu2017a,Tordesillas2019a,tordesillas2021mader}. In
the context of drone flying, \ac{sdf} has been utilized with
\ac{mpc} trajectory generation in unknown environments
\cite{Oleynikova2017voxblox}. Raw Lidar data has been used
in combination with \acp{rmp} showing impressively high
frequencies when computation is parallelized on GPU
\cite{Pantic2023obstacle}. Recent advances in sampling-based
\ac{mpc}, utilizing physics engines for collision avoidance
\cite{Pezzato2023sampling}, integrate obstacle collisions in
cost functions during trajectory rollouts. A similar
approach is seen in \cite{Sundaralingam2023curobo}. For
these approaches, the environment representation is implicit
through integration into the cost function, but requires
accurate modeling of the robot and the environment in the
physics engine used for rollout computation.

Though many methods demand manual robot representation, like
composing spheres, exploration of even more implicit
characterizations, such as learned \ac{sdf} like in
\cite{Liu2022regularized,Koptev2023neural}, is underway.


\section{Conclusion}
\label{sec:state_conclusion}

To summarize, path planning in robotics is a well-studied
problem. Especially in the early days of robotics, where
robots were mostly deployed in static environments, path
planning was a central problem. However, when exposed to
dynamic environments, path planning becomes insufficient due
to its inability to quickly react to changes in the
environment. This is especially true for mobile manipulation
systems, due to their high-dimensional configuration space,
\cite{Avanzini2018}. Environment representations are
crucially for all the planning methods described above.
Here, implicit representations have shown to be very useful
for cluttered, unknown and dynamic environments. However,
most research focuses on mobile robots, and manipulators
are yet to be explored in this context.


