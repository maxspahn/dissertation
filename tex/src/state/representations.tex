\section{Environment representations}
\label{sec:environment_representations}

All above described methods rely on some form of environment
representation. For example, the inequality constraints in
\ac{mpc} formulations responsible for collision avoidance
rely on a distance function between the robot and the
obstacles. Path planning methods, as described above, 
also require an environment representation, such that
collision checking can accept or reject new samples. In
geometric approaches, such as \ac{apf}, \ac{rmp} and
\ac{fabrics}, distance functions are also necessary.
For primitive shapes, such as spheres and boxes, distances
can be computed in closed form. When working in dynamic
environments, robots perceive their environment using vision
sensors, such as cameras, or range sensors, such as LiDAR.
Classifying the environment into primitive shapes is a
challenging task, especially when the environment is
cluttered and parts are occluded. A different approach to
collision avoidance relies on more implicit representations
of the environments --the most implicit would be raw sensor
data. In the following, we recall some works that focus on
such representations in the context of \ac{tg}.

\subsection{Implicit representations}
\label{sub:implicit_representations}

For trajectory planning using short-term optimization,
unoccupied space constraints limit robot movement, proven
useful for mobile robots in cluttered areas
\cite{Brito2019}. In drone flight, a similar concept
generates safe flight zones along a global path
\cite{Liu2017a,Tordesillas2019a,tordesillas2021mader}. In
the context of drone flying, \ac{sdf} has been utilized with
\ac{mpc} trajectory generation in unknown environments
\cite{Oleynikova2017voxblox}. Raw Lidar data has been used
in combination with \acp{rmp} showing impressively high
frequencies when computation is parallelized on GPU
\cite{Pantic2023obstacle}. Recent advances in sampling-based
\ac{mpc}, utilizing physics engines for collision avoidance
\cite{Pezzato2023sampling}, integrate obstacle collisions in
cost functions during trajectory rollouts. A similar
approach is seen in \cite{Sundaralingam2023curobo}. For
these approaches, the environment representation is implicit
through integration into the cost function, but requires
accurate modeling of the robot and the environment in the
physics engine used for rollout computation.

Though many methods demand manual robot representation, like
composing spheres, exploration of even more implicit
characterizations, such as learned \ac{sdf} like in
\cite{Liu2022regularized,Koptev2023neural}, is underway.
