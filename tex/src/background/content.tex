\chapter{Background} % (fold)
\label{cha:background}

This chapter provides the necessary background for the reader to understand the
contributions of this thesis. Firstly, we introduce
notations used throughout thesis.
After the introduction of some general notations, we recall the
\ac{mpc} framework in a general setting. Then, we state
relevant concepts of differential geometry that are required
in the aim of understanding \ac{fabrics}. This will lead us
the formal introduction of \ac{fabrics} which is the main
topic of this thesis. Overall, this chapter presens essential
background knowledge for the reader to understand the
remainder of the thesis. 

\section{Notations} % (fold)
\label{sec:notations}

Throughout this thesis, vectors are denoted in bold lowercase
letters, $\vec{x}$, matrices in capital, $\mat{M}$, and sets
in calligraphic uppercase, $\mathcal{M}$. In the context of
discrete-time systems, we denote the time step using a
subscript, e.g., $\vx_k$ is the state at time $k$. The
transpose of a matrix is denoted by a superscript $T$, and
the inverse by a superscript $-1$, the pseudo-inverse by a
superscript $\dagger$. Partial derivates are denoted by 
$\der{\vx}{\vy}$ in text block, or more explicitely as
\[\derf{\vx}{\vy}.\]
 $\norm{\vx}_M = \vx^TM\vx$
denotes the weighted squared norm. 


We denote $\q\in\Q\subset\Rn$ a configuration of the robot
with $n$ its degrees of freedom; \Q{} is the configuration
space of the generalized coordinates of the system.
Generally, $\q(t)$ defines the robot's configuration at time
$t$, so that \qdot, \qddot{} define the instantaneous
derivatives of the robot's configuration. Similarly, we
assume that there is a set of task variables
$\xj\in\Xj\subset\Rmj$ with variable dimension $m_j \leq n$.
The task manifold \Xj{} defines an arbitrary manifold of the
configuration space \Q{} in which a robotic task can be
represented. Further, we assume that there is a differential
map $\map_j:\Rn\rightarrow\Rmj$ that relates the
configuration space to the $j^{th}$ task space. For example,
when a task variable is defined as the end-effector
position, then $\map_j$ is the positional part of the
forward kinematics. On the other hand, if a task variable is
defined to be the joint position, then $\map_j$ is the
identity function. In the following, we drop the subscript
$j$ in most cases for readability when the context is clear.

In this work, we assume that \map{} is smooth and twice
differentiable so that the Jacobian is defined as
\begin{equation}
  \J = \derf{\q}{\map} \in \mathcal{R}^{m\times n}, 
\end{equation}
or $\J = \der{\q}{\map}$ for short.
Thus, we can write the total time derivatives of \x{} as
\begin{align}
  \xdot &= \J\qdot \\
  \xddot &= \J\qddot + \Jdot\qdot.
\end{align}

% section notations (end)

\section{Model Predictive Control} % (fold)
\label{sec:model_predictive_control}

Optimization methods, such as dynamic programming are
populare when addressing \ac{tg} problems that are
characterized by multiple equality and inequality
constraints. 
The optimization problem is composed of a cost function, 
a dynamic model of the system, and a set of constraints.

This thesis uses exclusively discrete time dynamics, such
that the dynamics of the system are given by 
\begin{equation}
  \z_{k+1} = f(\z_k, \u_k),
\end{equation}
where $\z_k \in \Rn$, $\u_k \in \Rm$ are the state and
control input vectors, respectively. 

Multiple objectives of the \ac{tg} problem are combined in
the cost function:
\begin{equation}
  J(\z,\u) = \sum_{k=0}^{N-1} J_k(\z_k, \u_k) + J_f(\z_N,\u_N),
\end{equation}
where $N$ is the prediction horizon, and $J_k$ and $J_f$ are
the stage and terminal cost, respectively.

The general \ac{mpc} problem is then formulated as
\begin{equation}
  \begin{aligned}
    \min_{\u} \quad & J(\z,\u) \\
    \text{subject to} \quad & \z_{k+1} = f(\z_k, \u_k), \quad k=0,\ldots,N-1 \\
    & g(\z_k, \u_k) \leq 0, \quad k=0,\ldots,N-1 \\
    & \z_k \in \Z, \quad \u_k \in \U, \quad k=0,\ldots,N \\
    & \z_0 = \z_{\textrm{init}},
  \end{aligned}
\end{equation}
where $\z_0$ is the initial state, $\Z$ and $\U$ are the
admissibe sets of states and control inputs, respectively.
Inequality constraints are denoted by $g(\z_k, \u_k)$.

At each time step $k$, the \ac{mpc} problem is solved to
find the sequence of control inputs $\u_k^*$ that minimizes
the cost function. The first control input of the optimal
sequence is then applied to the system. The remainder of the
sequence is shifted by one step to form the initial guess
for the next solution. This process is usually refered to as
receding horizon control.

\section{Differential Geometry} % (fold)
\label{sec:differential_geometry}

Differential geometry is the study of geometry using calculus.
It is a mathematical discipline that uses the techniques of
differential and integral calculus, as well as linear
algebra, to study problems in geometry. In the context of
this thesis, we recall some of the basic concepts of
differential geometry that are required to understand the
formalism of \ac{fabrics}.

\subsection{Manifolds} % (fold)
\label{sub:manifolds}

A manifold is a topological space that locally resembles
Euclidean space near each point. More precisely, a manifold
is a topological space $M$ suchjson that for each point $p\in
M$, there exists a neighborhood $U$ of $p$ that is
homeomorphic to an open subset of $\Rn$. The dimension of
the manifold is the dimension of the Euclidean space to
which it is locally homeomorphic.

A smooth manifold is a manifold that is equipped with a

\MS{Should I really go on like this? What is the purpose of
it?}


% subsection manifolds (end)

% section differential_geometry (end)

\section{Optimization Fabrics} % (fold)
\label{sec:optimization_fabrics}

In this section, we introduce the concept of optimization
fabrics. Optimization fabrics are a framework for motion
generation that is based on the theory of differential
geometry. The main idea is to design motion generation as a
second-order dynamical system. The trajectory generator is
defined by the differential equations as we will layout in
the following.

\subsection{Spectral Semi-Sprays} % (fold)
\label{sub:spectral_semi_sprays}

The 




$\M\xddot + \f = 0$, where $\M(\x,\xdot)$ and $\f(\x,\xdot)$
are functions of position and velocity. Besides, \M{} is
symmetric and invertible. 

\subsection{Spectral semi-sprays}%
\label{sub:spectral_semi_sprays}

Inspired by simple mechanics (e.g., the simple pendulum),
the framework of optimization fabrics designs motion
generation as second-order dynamical systems $\xddot =
\pi(\x,\xdot)$~\cite{Cheng2020,Ratliff2020}. The
trajectory generator is defined by the differential equation
$\M\xddot + \f = 0$, where $\M(\x,\xdot)$ and $\f(\x,\xdot)$
are functions of position and velocity. Besides, \M{} is
symmetric and invertible. Such systems $\Spec = \spec$ are
known as \textit{spectral semi-sprays}, or \textit{specs}
for short.  When the space of the task variable is clear
from the context, we drop the subscript.  Then, the
trajectory is computed as the solution to the system
$\xddot=-\M^{-1}\f$.

\subsection{Operations on specs}%
\label{sub:operations_on_specs}
Next, the two fundamental operations for specs, transformation between spaces and
summation, are introduced.

Given a differential map $\map: \Q\rightarrow\X$ and a spec \spec{}, the \textit{pullback}
is defined as 
\begin{equation}
  \pull{\map}{\spec} = {\left(\Jt\M\J, \Jt(\f+\M\Jdot\qdot)\right)}_{Q}.
\end{equation}
The pullback allows converting between two distinct manifolds (e.g. a spec could be 
defined in the robot's workspace and being pulled into the robot's configuration space using
the pullback with \map{} being the forward kinematics).

For two specs, $\Spec_1 = {\left(\M_1,\f_1\right)}_{\X}$ and 
$\Spec_2 = {\left(\M_2,\f_2\right)}_{\X}$, their \textit{summation} is defined by:
\begin{equation}
  \Spec_1 + \Spec_2 = {\left(\M_1 + \M_2, \f_1 + \f_2\right)}_{\X}.
\end{equation}

\subsection{Optimization fabrics}%
\label{sub:optimization_fabrics}

Optimization fabrics form a special class of specs, and thus they inherit their properties,
specifically the previously defined operations of \textit{summation} and \textit{pullback}.
First, let us introduce a finite and differentiable potential
function $\forc(\x)$ defined in a task manifold \X{}. 
Then, the modified spec $\Spec_{\forc} = \left(\M,\f + \der{\x}{\forc}\right)$
is called the \textit{forced variant} of $\Spec = \spec$.
Only if the trajectory $\x(t)$ generated by the forced spec converges to the minimum of \forc{}, 
the spec is said to form an \textit{optimization fabric}.
When the spec only converges to the minimum when equipped with a damping term,
$\left(\M,\f + \der{\x}{\forc} + \mat{B}\xdot\right)$, 
it forms a \textit{frictionless fabric}~\cite[Definition 4.4]{Ratliff2020}. 
Note that the mechanical system of a pendulum forms a frictionless fabric, as it optimizes
the potential function defined by gravity when being damped (i.e., it eventually comes to
rest at the configuration with minimal potential energy)
%, see Appendix
%\ref{app:sub:motion_as_second_order_systems} for a detailed discussion on the single
%pendulum).

In the following, methods to construct optimization fabrics, or \textit{fabrics} for
short, are summarized: the definitions of conservative fabrics and energization are
introduced.

\subsection{Conservative fabrics and energization}%
\label{sub:conservative_fabrics_and_energization}

While the previous subsection defined what criteria are required for a spec to form
an optimization fabric, the theory on conservative fabrics and energization 
offers a simple way of generating such special specs. As a full summary of the theory
on optimization fabrics and their construction is out of scope here, this
subsection only provides an outline of the theory and the reader is referred to 
\cite{Ratliff2021,Wyk2022} for detailed derivations.

In the context of fabrics, the term \textit{energy} describes a scalar quantity that
changes as the system evolves over time.  Although this quantity has a physical meaning in
natural systems (e.g., kinetic energy), it can be arbitrarily defined for motion generation.
Generally, specs and optimization fabrics do not conserve an energy, but when they do, we
call them \textit{conservative specs}.  A stationary Lagrangian~\cite[Definition
4.11]{Ratliff2020} is one definition for an energy for which the corresponding spec, known
as the Lagrangian spec $\Spec_{\le} = \left(\Me,\fe\right)$, is obtained by applying the
Euler-Lagrange equations. Importantly, Lagrangian specs conserve energy and do thus belong
to the class of \textit{conservative specs}.  It was proven that an unbiased
(\cite[Definition~4.11]{Ratliff2020}) Lagrangian spec forms a frictionless
fabric~\cite[Proposition~4.18]{Ratliff2020}. Such fabrics are analogously called
\textit{conservative fabrics}.  There are two classes of conservative fabrics: Lagrangian
fabrics (i.e., the defining energy is a Lagrangian) and the more specific subclass of
Finsler fabrics (i.e., the
defining energy is a Finsler structure~\cite[Definition 5.4]{Ratliff2020}). 

The operation of \textit{energization} transforms a given differential equation into
a conservative spec.
Specifically, given an unbiased energy Lagrangian \le{} with boundary conforming
\Me{}~\cite[Definition~4.6]{Ratliff2020} and
lower bounded energy \he{}, an unbiased spec of form $\Spec_{\vec{h}} = (\mat{I},\vec{h})$
is transformed into a frictionless fabric using energization as
\begin{equation}
  \begin{split}
  S_{\vec{h}}^{\le} &= \text{energize}_{\le}\{S_{\vec{h}}\} \\
    &= (\Me, \fe + \Pe[\Me\vec{h} - \fe]), 
  \end{split}
\end{equation}
where $\Pe = \Me\left(\Me^{-1} - \frac{\xdot\xdot^T}{\xdot^T\Me\xdot}\right)$ is an
orthogonal projector.
Energized specs maintain the energy of the Lagrangian and generally change
the trajectory of the underlying spec $\Spec_{\vec{h}}$.
However, if 
\begin{enumerate}
  \item $\Spec_{\vec{h}} = (\mat{I},\vec{h})$ is homogeneous of degree 2,
    \begin{equation}\vec{h}(\x, \alpha\xdot) = \alpha^2\vec{h}(\x, \xdot)\label{eq:homogeneous}\end{equation}
    and
  \item the energizing Lagrangian is a Finsler structure, 
\end{enumerate}
the resulting energized spec forms a frictionless fabric for which the trajectory matches
the original trajectory of $\Spec_{\vec{h}}$. We refer to energized fabrics with that
property as \textit{geometric fabrics}. Geometric fabrics form the building blocks for
motion generation with optimization fabrics.
Practically, energization equips the individual components of the planning problem
with a metric when being combined with other components.


\subsection{Experimental results fabrics}%
\label{sub:experimental_results_fabrics}

The theory explained above was tested on several simple kinematic chains
in~\cite{Ratliff2020,Ratliff2021}. As fabrics design motion as a summation of several
differential equations, each representing a specific constraint to the motion, it is
possible to sequentially design motion~\cite{Ratliff2020}. This procedure allows to
carefully tune individual components without harming the others. The application to a
planar arm in a goal-reaching setup was successfully tested in~\cite{Ratliff2020}. Here,
the authors illustrated how the resulting motion can be modified arbitrarily by the user
by adding additional constraints or preferences.

Although important concepts and findings on optimization fabrics were summarized in this
section, we refer to~\cite{Ratliff2020} for a more in-depth presentation of optimization
fabrics. In the following, we generalize the framework of optimization fabrics to dynamic settings.

\section{Background-ICRA23}
\label{sec:optimization_fabrics}
%
In this section, we very briefly introduce the concepts required for trajectory
generation with optimization fabrics. For a more in-depth introduction to
optimization fabrics and its foundations in differential geometry, the reader
is referred to \cite{Ratliff2020,Spahn2022,Wyk2022}.
%
\subsection{Configurations and task variables}%
\label{sub:configurations_and_task_variables}
%
We denote $\q\in\Q\subset\Rn$ a
configuration of the robot with $n$ its degrees of freedom;
\Q{} is the configuration space of the generalized coordinates
of the system. Generally, $\q(t)$ defines the robot's configuration at time $t$, so that 
\qdot, \qddot{} define the instantaneous derivatives of the robot's configuration.
Similarly, we assume
that there is a set of task variables $\xj\in\Xj\subset\Rmj$ with variable dimension
$m_j \leq n$. The task space \Xj{} defines an arbitrary manifold of the configuration
space \Q{} in which a robotic task can be represented. 
Further, we assume that there is a differential map
$\map_j:\Rn\rightarrow\Rmj$ that relates the configuration space to the $j^{th}$ task
space. For example, when a task variable is defined as the end-effector position, then
$\map_j$ is the positional part of the forward kinematics. On the other hand, if a task
variable is defined to be the joint position, then $\map_j$ is the identity function. 
In the following, we drop the subscript $j$ in most cases for readability when the
context is clear.

We assume that \map{} is in $\mathcal{C}^1$ so that the Jacobian is
defined as
\begin{equation}
  \J = \derf{\q}{\map} \in \mathcal{R}^{m\times n}, 
\end{equation}
or $\J = \der{\q}{\map}$ for short.
Thus, we can write the total time derivatives of \x{} as
$\xdot = \J\qdot$ and $\xddot = \J\qddot + \Jdot\qdot$.
%
\subsection{Spectral semi-sprays}%
\label{sub:semi_spectral_sprays}
%
Inspired by simple mechanics (e.g., the simple pendulum), the framework of optimization
fabrics designs motion policies as second-order dynamical
systems $\xddot = \pi(\x,\xdot)$~\cite{Cheng2020,Ratliff2020}.
The motion policy is defined by the differential equation
$\M\xddot + \f = 0$, where $\M(\x,\xdot)$ and $\f(\x,\xdot)$ are functions of position and
velocity. Besides, \M{} is symmetric and invertible. We denote such systems as $\Spec = \spec$ and
refer to them as \textit{spectral semi-sprays}, or \textit{specs} for short.  When the space of
the task variable is clear from the context, we drop the subscript. 

\subsection{Operations on specs}%
\label{sub:operations_on_specs}
%
%
Complex trajectory generation is composed of multiple components, such as collision avoidance, joint limits
avoidance, etc. The power of optimization fabrics lies in the metric-weighted sum to
combine multiple components from different manifolds.
These operations are derived from operations on specs and are briefly recalled here.

Given a differential map $\map: \Q\rightarrow\X$ and a spec \spec{}, the \textit{pullback}
is defined as 
\begin{equation}
  \pull{\map}{\spec} = {\left(\Jt\M\J, \Jt(\f+\Jdot\qdot)\right)}_{Q}.
\end{equation}
The pullback allows converting between two distinct manifolds (e.g. a spec could be 
defined in the robot's workspace and pulled into the robot's configuration space using
the pullback with \map{} being the forward kinematics).

For two specs, $\Spec_1 = {\left(\M_1,\f_1\right)}_{\X}$ and 
$\Spec_2 = {\left(\M_2,\f_2\right)}_{\X}$, their \textit{summation} is defined by:
\begin{equation}
  \Spec_1 + \Spec_2 = {\left(\M_1 + \M_2, \f_1 + \f_2\right)}_{\X}.
  \label{eq:specs_summation}
\end{equation}
%

Additionally, a spec can be \textit{energized} by a Lagrangian energy. Effectively, 
this equips the spec with a metric.
Specifically, given a spec of form $\Spec_{\vec{h}} = (\mat{I},\vec{h})$ and 
an energy Lagrangian \le{} with the derived equations of motion $\M_{\le}\xddot + \f_{\le} =0$, 
we can define the operation
\begin{equation}
  \begin{split}
  S_{\vec{h}}^{\le} &= \text{energize}_{\le}\{S_{\vec{h}}\} \\
    &= (\Me, \fe + \Pe[\Me\vec{h} - \fe]), 
  \end{split}
  \label{eq:energization}
\end{equation}
where $\Pe = \Me\left(\Me^{-1} - \frac{\xdot\xdot^T}{\xdot^T\Me\xdot}\right)$ is an
orthogonal projector. The resulting spec is an \textit{energized spec} and 
we call the operation \textit{energization}.

With spectral semi-sprays and the presented operations,
avoidance behavior, such as joint limit avoidance, collision
avoidance or self-collision avoidance, can be realized.

\subsection{Optimization fabrics}%
\label{sub:optimization_fabrics}
%
In the previous subsection, we explained how different avoidance behaviors can
be combined. Spectral semi-sprays can additionally be \textit{forced} by a
potential, denoted as the \textit{forced variant} of form $\Spec_{\forc} =
\left(\M,\f + \der{\x}{\forc}\right)$. This forcing term clearly changes the
behavior of the system. Optimization fabrics introduce construction rules to
make sure that the solution path $\x(t)$ of $\Spec_{\forc}$ converges towards
the minimum of $\forc(\x)$. Then, the potential is designed in such a way that
its minimum represents a goal state of the motion planning problem.

First, the initial spec that represents an avoidance component
is written in the form $\xddot + \vec{h}(\x,\xdot) = 0$,
such that $\vec{h}$ is \textit{homogeneous of degree 2}:
$\vec{h}(\x,\alpha\xdot) = \alpha^2\vec{h}(\x,\xdot)$ 
(\textbf{Creation}). 
Secondly, the geometry is
energized (\cref{eq:energization})
with a Finsler structure~\cite[Definition 5.4]{Ratliff2020} (\textbf{Energization}).
The property of homogeneity of degree 2 and the energization with the Finsler structure
guarantees, according to ~\cite[Theorem 4.29]{Ratliff2020}, that the energized spec
forms a \textit{frictionless fabric}.
A frictionless fabric is defined to optimize the forcing potential \forc{} when being
damped by a positive definite damping term~\cite[Definition 4.4]{Ratliff2020}.
Thirdly, all avoidance components are combined in the configuration space
of the robot using the pullback and summation operation (\textbf{Combination}).
Note, that both operations are closed 
under the algebra designed by these operations, i.e. every pulled optimization fabric or the sum
of two optimization fabrics is, itself, an optimization fabric.
In the last step, the combined spec is forced by the potential \forc{} with the desired minimum and
damped with a positive definite damping term (\textbf{Forcing}).
This resulting system of form $\M\qddot + \f(\q, \qdot) + \der{\q}{\forc} + \beta\qdot = 0$ is solved to 
obtain the trajectory generation policy in acceleration form $\qddot = \pi(\q,\qdot)$.

