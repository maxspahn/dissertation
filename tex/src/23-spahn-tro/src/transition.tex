\begin{abstract}
The implementation of \ac{mpc} demonstrated its capability to
generate whole-body trajectories for a mobile manipulator efficiently. Employing
free-space decomposition on individual robot links effectively reduced
computational costs.
However, solving the optimization problem remained
computationally expensive, limiting control frequency and yielding non-smooth
trajectories.
Consequently, this chapter marks a shift away from
optimization-based methods towards a geometric approach known as \ac{fabrics}.
This method allows the encoding of trajectory generation in ordinary
differential equations of second order, presenting an alternative to \ac{mpc}
that offers much faster computation cycles and more reactive motion.
Although geometric fabrics have proven
applicable to real-time trajectory generation, previous works are limited
to simple, static environments that are free of local minima and global path
planning is thus not required.
This chapter introduces the concept of \ac{df} to overcome these limitations.
Moreover, extensive comparisons between \ac{mpc} and \ac{sf} are conducted to
substantiate the effectiveness of geometric encoding in trajectory generation.
Finally, real-world experiments are conducted, involving both a robot arm
and a mobile manipulator, to validate the proposed methodologies.
\end{abstract}
