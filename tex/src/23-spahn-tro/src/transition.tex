\begin{abstract}
The earlier presented implementation of \acl{mpc} demonstrated its capability to
generate whole-body trajectories for a mobile manipulator efficiently. Employing
free-space decomposition on individual robot links effectively reduced
computational costs.
However, solving the optimization problem remained
computationally expensive, limiting control frequency and yielding non-smooth
trajectories.
Consequently, this chapter marks a shift away from
optimization-based methods towards a geometric approach known as \acl{fabrics}
This method allows the encoding of trajectory generation in ordinary
differential equations of second order, presenting an alternative that addresses
the aforementioned limitations. Although geometric fabrics have proven
applicable to real-time trajectory generation, presented works are limited
to static environments, where global path planning is not required.
This chapter introduces the concept of \acl{df} to overcome these limitations.
Moreover, extensive comparisons between \ac{mpc} and \acl{sf} are conducted to
substantiate the effectiveness of geometric encoding in trajectory generation.
Finally, real-world experiments are conducted, involving both the manipulator
and the mobile manipulator, to validate the proposed methodologies.
\end{abstract}
