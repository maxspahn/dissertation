

\subsection{Considerations and vision}
\label{sec:discussion_vision}

After having discussed the advantages and limitations of
methods related to the chapters of this thesis, I also want
to give a broader perspective on robotics. In particular, I
want to shed light on some reasons why robots have not
entered human shared environments in a more widespread
manner. To make this section a bit more lively, I shall ask
some questions, that are often asked by the public, and
try to answer them in a more philosophical way.


\textit{Why does the robot move so slow?}

This is the question that every roboticist
-- especially those working on \ac{tg} --
has to endure when showing their work to the public. While most, including the
author, are quick to point out that the robot moves slow to ensure safety, the
true reason is more complex. Execution speed is dependent on
(a) the ability to move fast,
(b) the ability to execute trajectories accurately when moving fast,
(c) the ability to make state estimates fast,
(d) the ability to compute trajectories fast,
(e) the effect of speed of motion on the environment,
and, of course,  (f) safety considerations.
Let me treat these
points in turns:
% enumeration in alphabetical order (a) - (f)
\begin{enumerate}[(a)]
  \item When a robot's motors are not powerful enough to
    move fast, the robot will move slow. This is obvious and
    should not be used as an excuse, because all robots used
    in this thesis are capable of moving much faster than
    shown in the experiments. 
  \item Moving fast however is not sufficient as accuracy
    is important in manipulation tasks. It turns out
    that modern hardware is capable of speeding up
    moving between waypoints defined in the configuration
    space. After all,
    robots in cages are moving extremely fast with
    super-human accuracy, so this cannot be the reason for
    slow motion either.
  \item When deployed in human-shared environments however,
    robots cannot simply play back sequences of waypoints in
    an open-loop manner. Instead, the environment has to be
    constantly perceived and the state of the robot must be
    estimated at the same speed. It is difficult
    to achieve required update frequency with computer
    vision, but even state estimates related to the robot
    itself fail to be accurate at high speeds. For example,
    the joint velocities of the \panda{} used primarily
    in this thesis are not accurate at high speeds. That
    proves to be highly problematic when using reactive
    \ac{tg} methods. In this thesis, the problem was usually
    addressed by low-pass filtering the joint velocities
    which on the flip side introduces a delay of the signal.
    Either the estimates must be improved without
    introducing delays or \ac{tg} methods must be robust
    against noise estimates. Inaccurate, slow or delayed
    sensing seems an important reason for slow motion in
    human-shared environments.
  \item The ability to compute trajectories fast is
    discussed in \cref{sec:discussion_fabrics_or_mpc} and I
    believe that \ac{fabrics} is a good approach to
    address slow computations causing slow-moving robots.
    However, more widely used methods do not offer this
    speed and are therefore contributing to slow motion.
  \item What is often under-appreciated by the public is
    that speeding up motions has a direct effect on the
    interactions with the environment. For example, when
    grasping objects with a vacuum gripper, as done in 
    \cref{cha:rss_24}, the motion must be slow when
    approaching an item to avoid it tipping over. This
    logical connection is often considered trivial by the
    public, but is very difficult to encode in a robot's
    behavior. This leads to a tradeoff between speed of
    motion and success rate, where the latter is usually
    favored.
  \item And finally, speed of motion and safety are in direct
    conflict. We put fast robots in cages because their
    kinetic energy is dangerous to humans. Having the same
    kind of robots in human-shared environments naturally 
    results in slow motions to compensate for the lack of
    the cage as a safety measure. Unless, we start relying
    on soft robots to imitate human hardware, this conflict
    seems hard to impossible to resolve.
\end{enumerate}

\textit{Why does the robot choose this highly complicated
way of moving?}

This question often arises when a robot seems to take an
unconventional path to reach its destination. The issue
hinges on the notion of what is considered \textit{natural},
which essentially boils down to what humans perceive as the
best route. In the context of robotics, the optimal
path is typically defined as the shortest path in the
configuration space, which poses significant challenges due
to its high dimensionality and the constraints imposed by
obstacles. Consequently, finding this shortest path becomes
a difficult task, often addressed through sampling-based
methods like Rapidly-exploring Random Trees (RRT), which
iteratively sample configurations until a viable sequence of
configurations leading to the goal is discovered
\cite{Karaman2011}. However, this approach only guarantees
the shortest connection between subsequent samples, not the
entire sequence, often resulting in abrupt and jerky
motions. Despite efforts to refine these paths, such as
through smoothing techniques
\cite{siegwart2011introduction}, the fundamental challenge
lies in the nature of planning within the configuration
space, which tends to yield paths that appear unconventional
to human observers. In contrast, the methods outlined in
this thesis avoid long-term planning within the
configuration space, instead relying on purely reacive \ac{tg}
and global path planning in the workspace, which generally
yields paths that appear more natural. Specifically,
techniques like shaping the configuration manifold seem to
resemble to human path-planning processes
\cite{klein2022riemannian}.

\textit{Why does the robot fail at this simple task?}

This is not directly a question of the \ac{tg} problem at
hand, but it is often intertangled with it. It touches upon
task and motion planning, a widely studied field of robotics
these days, see \cite{garrett2021integrated} for a recent
survey. To answer this question, one has to appreciate the
ease of humans to combine several sensors, e.g. vision,
feel, sound, etc. with general understanding of physics and
highly precise, yet compliant actuation. Some works suggest
that the sensors of robots are fundamentally limiting the set of
tasks a robot could do \cite{majumdar2023fundamental}.
Additionally, our cognitive abilities to understand failure
cases, make long term plans and to transfer knowledge are
not met by robots. Until great advancements are made, the
public will continue wondering why robots fail at simple
tasks.

While our modern societies are desperate for automation in human-shared
environments, it is often disappointing to observe the current state of the art
in robotics. From an intellectual point of view however, this is truly exciting,
as the opportunities to shape robotics seem endless. And, from a philosophical
point of view you may ask: \textit{Isn't it quite reassuring that human
abilities are still superior to those of robots when it
comes to human-shared environments?}.
