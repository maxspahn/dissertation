\chapter{Conclusion and discussion}
\label{cha:conclusion_and_discussion}

The final chapter of this thesis summarizes the contributions and discusses the
main results. Besides, research questions for the following years and raised and
potential approaches are layed out. This chapter ends with a vision on the
future of robotics in human-shared environments.

\section{Conclusion}
\label{sec:conclusion}

This thesis mainly contributed to geometric interpretation of \ac{tg} for manipulators
and mobile manipulators. Specifically, we extended the framework of \ac{fabrics}
to more dynamic environments, proposed a symbolic implementation to allow for
fast parameter tuning at runtime, and demonstrated its effectiveness in
a real demonstration case. Along the way, we showed how \ac{mpc} can be used for
mobile manipulators and why geometric approaches tend to outperform
optimization-based methods for these systems. All results were validate
extensively in simulation and verified in real-world experiments.
In the following, we summarize the main contributions of this thesis.

\subsection{MPC for mobile manipulators}
\label{sec:conclusion_mpc}

\cref{icra_spahn_21} extended the widely used method of \ac{fsd} to mobile
manipulators which lead to several improvements. First, it allowed to control
the system in a coupled way reducing overall execution time, similar to what had
been done in \cite{Avanzini2018} with dynamic weighing. Second, it reduced the
required accuaracy of the perception pipeline as raw pointclouds or occupancy
grids can be used. Third, we showed that computationally scaling in the number
of obstacles can be prevented. In the context of this thesis, we additionally
showed that geomteric approaches tend to outperform \ac{mpc} formulations
despite weaker theoretical guarantees.

\subsection{Dynamic fabrics}
\label{sec:conclusion_dynamic_fabrics}

\cref{tro_spahn_23} generalized the framework of \ac{fabrics} to dynamic
environments. We showed that \ac{fabrics} can be formulated on moving reference
frames, in such a way that trajectory following and collision avoidance with
dynamic obstacles can be combined with other, static components to the \ac{tg}
problem. Specifically, the \textit{dynamic pull-back}
operator brings a dynamically defined component into the static manifold of the
configuration space while making the motion of the frame a runtime variable of
the root geometry. This chapter shows that reference velocity integration is
required to handle moving obstacles. Additionally, integration of global path
planning in arbitrary manifolds is possible to the generalization. In
particular, this chapter shows that convergence to reference paths can be 
guaranteed. Finally, this chapter presents results on path following with a
mobile manipulator and collision avoidance with moving humans in the workspace.
\MS{What about the extension to non-holonomic systems?}


\subsection{Symbolic fabrics}
\label{sec:conclusion_symbolic_fabrics}

\cref{icra_spahn_23} highlighted the importance of the existence of a
closed-forme solution to \ac{tg} when formulated as a geometric problem.
This property is exploited in a purely symbolic implementation of \ac{fabrics}
which allows for fast parameter tuning at runtime. This chapter addationally
proposes to use a Bayesian method to tune parameters to \ac{fabrics} at runtime.
The experiments showed that expert-level can be reached within 50 iterations.
Additionally, the results reveal that parameter sets can be transfered between
similar embodiments. Finally, qualitative results on a real robot are presented
confirming the applicability of the approach.

\subsection{Real-world demonstration}
\label{sec:conclusion_real_world}

\cref{rss_spahn_24} demonstrated that \ac{fabrics} are not a mere theoretical
framework but can be used in real-world manipulation demos. In the combination
with a modern decision making framework, and state-of-the-art perception, 
we demonstrated how \ac{fabrics} can be used to generate trajectories in complex
environments with high-dimensional robots. Besides, it was shown that different
paradigms from \ac{tg} in manipulation, such as learning-based methods and
redundancy resolution, can be easily integrated into the framework.






\section{Discussion}
\label{sec:discussion}

Here, we should discuss limitations and future works.
Geometric approaches: Potential ad limitations.




