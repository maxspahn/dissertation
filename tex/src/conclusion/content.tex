\chapter{Conclusion and discussion}
\label{cha:conclusion_and_discussion}

The final chapter of this thesis summarizes the contributions and discusses the
main results. Besides, research questions for the following years and raised and
potential approaches are layed out. This chapter ends with a vision on the
future of robotics in human-shared environments.

\newpage

\section{Conclusion}
\label{sec:conclusion}

% Reference back to introduction
This thesis was stated in the context of labor shortage in the global North and
the need for further automation to combat demographic change. In particular, we
focused on the problem of \ac{tg} for mobile manipulators in human-shared
environments. This thesis proposed several \ac{tg} methods to allow the safe
deployment of mobile manipulators outside their classical cages.
% Summary of the contributions
This thesis mainly contributed to geometric interpretation of \ac{tg} for
manipulators and mobile manipulators. Specifically, we extended the framework of
\ac{fabrics} to more dynamic environments, proposed a symbolic implementation to
allow for fast parameter tuning at runtime, and demonstrated its effectiveness
in a real demonstration case. Along the way, we showed how \ac{mpc} can be used
for mobile manipulators and why geometric approaches tend to outperform
optimization-based methods for these systems. All results were validated
extensively in simulation and verified in real-world experiments. In the
following, we summarize the main contributions of this thesis.

\subsection{MPC for mobile manipulators}
\label{sec:conclusion_mpc}

\cref{icra_spahn_21} extended the widely used method of \ac{fsd} to mobile
manipulators which lead to several improvements. First, it allowed to control
the system in a coupled way reducing overall execution time, similar to what had
been done in \cite{Avanzini2018} with dynamic weighing. Second, it reduced the
required accuaracy of the perception pipeline as raw pointclouds or occupancy
grids can be used. Third, we showed that computationally scaling in the number
of obstacles can be prevented. To relate to the other chapters of this thesis,
we additionally showed that geomteric approaches tend to outperform \ac{mpc}
formulations despite weaker theoretical guarantees.

\subsection{Dynamic fabrics}
\label{sec:conclusion_dynamic_fabrics}

\cref{tro_spahn_23} generalized the framework of \ac{fabrics} to dynamic
environments. We showed that \ac{fabrics} can be formulated on moving reference
frames, in such a way that trajectory following and collision avoidance with
dynamic obstacles can be combined with other, static components to the \ac{tg}
problem. Specifically, the \textit{dynamic pull-back}
operator brings a dynamically defined component into the static manifold of the
configuration space while making the motion of the frame a runtime variable of
the root geometry. This chapter shows that reference velocity integration is
required to handle moving obstacles. Additionally, integration of global path
planning in arbitrary manifolds is possible to the generalization. In
particular, this chapter shows that convergence to reference paths can be 
guaranteed. Finally, this chapter presents results on path following with a
mobile manipulator and collision avoidance with moving humans in the workspace.
\MS{What about the extension to non-holonomic systems?}


\subsection{Symbolic fabrics}
\label{sec:conclusion_symbolic_fabrics}

\cref{icra_spahn_23} highlighted the importance of the existence of a
closed-forme solution to \ac{tg} when formulated as a geometric problem.
This property is exploited in a purely symbolic implementation of \ac{fabrics}
which allows for fast parameter tuning. This chapter addationally
proposes to use a Bayesian optimization to tune parameters to \ac{fabrics}.
The experiments showed that expert-level can be reached within 50 iterations.
Additionally, the results reveal that parameter sets can be transfered between
similar embodiments. Finally, qualitative results on a real robot are presented
confirming the applicability of the approach.

\subsection{Implicit environment representations}
\label{sec:conclusion_implicit}

\MS{Depending on the whether this chapter gets included.}

\subsection{Real-world demonstration}
\label{sec:conclusion_real_world}

\cref{rss_spahn_24} demonstrated that \ac{fabrics} are not a mere theoretical
framework but can be used in real-world manipulation demos. In the combination
with a modern decision making framework, and state-of-the-art perception, 
we demonstrated how \ac{fabrics} can be used to generate trajectories in complex
environments with high-dimensional robots. Besides, it was shown that different
paradigms from \ac{tg} in manipulation, such as learning-based methods and
redundancy resolution, can be easily integrated into the framework.


\section{Discussion}
\label{sec:discussion}

This thesis focused on different methods for \ac{tg} that are capable to
generate motion for mobile manipulators in human-shared environments. That
included the adaption to the widely used methods of \ac{mpc} to mobile
manipulation. The main part of this thesis was dedicated to a geometric
interpretation of \ac{tg}. Although several advancements for bringing mobile
manipulators to human-shared environments in a safe manner were made, we acknowledge
that seeing robots in our daily lives will still be a rare sight for the years
to come. In the following, we discuss the limitations of the proposed methods
and highlight important missing pieces to unlock the full potential of mobile
manipulators in human-shared environments.

\subsection{Fabrics or MPC?}
\label{sec:discussion_fabrics_or_mpc}

Most readers are likely familar with \acl{mpc}, fewer may have seen its application
to for example autonomous driving or mobile manipulation, and some 
experienced roboticists may have be aware of the strengths and weaknesses of
\ac{mpc}. In contrast, few reader have heard of \acf{fabrics} before and are 
quick to make a connection between both methods due to the word
\textit{optimization}. However, the methods are fundamentally different.
and it is important to analyse the differences and similarities between
these methods in an understandable way.
% MPC
\ac{mpc} is centered around the
formulation of a constrained optimization problem. The constraints are
the dynamics of the system and avoidance behaviors, such as collision avoidance
or joint limit avoidance. The problem is solved up to a defined optimality 
criteria. In general, the optimization problem is non-linear and non-convex.
% fabrics
In contrast, \ac{fabrics} is a purely geometric method. It is based on the
formulation of a smooth manifold in the configuration space of the robot.
Then, the solution to the \ac{tg} problem is solution to the geodesic equation
where the manifolds metric is non-Riemannian in general. Importantly, the entire
problem is encoded in the manifolds metric by iteratively adding avoidance
behaviors, such as collision avoidance or joint limit avoidance. As the
trajectory is consequence of solving the geodesic equation in each time step, a
closed-form solution is available for explicit behaviors as discussed in this
thesis.
% transition
This fundamentally different view on \ac{tg} has severe implications on
practical implementations.

\paragraph{Reactivity}
\label{par:discussion_reactivity}

\paragraph{Motion design}
\label{par:motion_design}


That has the advantage of being able to solve the problem in closed-form, which
is generally not possible for \ac{mpc} formulations. However, all behaviors must
be encoded on smooth manifolds of the configuration space, making the method
less flexible to arbitrary constraints, that may be integrated into an \ac{mpc}
formulation.

\subsection{Limitations}
\label{sec:discussion_limitations}

\textit{Why moves the robot so slow?} This is the question that every roboticist
-- especially those working on \ac{tg} --
has to endure when showing their work to the public. While most, including the
author, are quick to point out that the robot moves slow to ensure safety, the
true reason is more complex.

\paragraph{Hardware}

\paragraph{Safety}

\textit{Why is does the robot fail at this simple task?} This is not directly a
question of the \ac{tg} problem at hand, but it is often intertangled with it.
It touches upon task and motion planning, a widely studied field of robotics
these days, see \cite{garrett2021integrated} for a recent survey. Especially,
when 

\subsection{Vision for trajectory generation}
\label{sec:discussion_vision}

While our modern societies are desperate for automation in human-shared
environments, it is often disappointing to observe the current state of the art
in robotics. From an intelectual point of view however, this is truely exciting,
as the opportunities to shape robotics seem endless. And, from an philosophical
point of view you may ask: \texitit{Isn't it quite reassuring that human
abilities are still superior to those of robots when it comes to dynamic
environments?}.
