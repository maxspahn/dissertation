\section{Related Works}%
\label{sec:related_works}

Past works devoted to motion planning for mobile manipulators can be divided
into two main categories: optimization-based and sampling-based methods
\cite{LaValle2006,Mukadam2017}. Sampling-based methods, such as rapidly
exploring random trees (RRT's) \cite{Webb2013,Kleinbort2019,Kuffner2000} and
probabilistic roadmaps (PRM's) \cite{Hsu2002,Faust2017} are highly efficient at
generating paths for systems with high-degree of freedom. However, these
approaches consider static environments requiring a complete replanning if the
conditions change and hence, are not applicable for dynamic environments
\cite{Avanzini2018}.

In contrast, trajectory planning methods focus on executing generated paths
while avoiding collisions in dynamic environments. collisions developing motion
planning methods for mobile manipulators, most approaches either tackle the
motion planning problem for the robot's base or the robot's arm. Pioneer works
on motion planning methods for statically mounted manipulators employed
potential fields for collision avoidance \cite{Khatib1985}. Building on the
previous, \cite{Haddadin2011} introduced the Circular Field method to address
dynamic collision avoidance. Finally, to ensure collision avoidance for the
end-effector when grasping a moving obstacle, \cite{Du2018} employed a
repulsive vector.

\subsection{Collision Avoidance For Mobile Robots}
The dynamic window approach \cite{Fox1997} and its new variant proposed in
\cite{Zhang2019} have proven to be efficient in generating smooth trajectories
for mobile robots in static and dynamic environments. To navigate among
pedestrians, \cite{Ferrer2013} introduced the Social Forces model imitating
the human navigation behavior and using it as navigation policy for the
robot..  Yet, Social Forces and its variants rely on handcrafted functions
limiting their ability to handle more complex navigation scenarios. To deal
with a large number of agents, \textit{ORCA} was proposed in
\cite{VanDenBerg2011} and later extended for non\hyp{}holonomic bases in
\cite{Alonso-Mora2012a}. However, these approaches demonstrate highly reactive
behaviors because they only consider one step look-ahead predictions. MPC
schemes were proposed for mobile robots and autonomous vehicles in
\cite{Brito2019} and \cite{Schwarting2018} allowing to optimize over a
prediction horizon and avoid, in advance, dynamic obstacles. To enable coupled
control of a mobile manipulator, collision avoidance must be performed in the
3D space which is usually not necessary for ground vehicles. Several 3D MPC
formulation were proposed for drones to enable safe motion through cluttered
environments \cite{Tordesillas2019,Liu2017}.

\subsection{Collision Avoidance For Mobile Manipulators}
Despite abundant research in trajectory planning for mobile robots and robotic
arms, few works focused on coupling both systems' control. It was shown that
coupling the base and the robotic arm motion leads to a considerable reduction
of total operational time and smoother motions \cite{Thakar2018, Thakar2019}.
Nevertheless, these methods were designed for static environments and did not
allow real-time collision avoidance. Furthermore, trajectory planning for the
coupled system is a precondition for effective interactive navigation, including
opening doors \cite{Jain2009, Chitta2010} or moving obstacles out of the way
\cite{Li2019}.

In the context of mobile manipulation, less research focused on collision
avoidance in dynamic environments, including changing scenes and moving
obstacles. A real-time controller using MPC was presented in \cite{Ide2011}, in
which either a holonomic or a non\hyp{}holonomic base was combined with a
two-degree-of-freedom robotic arm mounted onto the base. Although hardware
constraints were respected, no collision avoidance was considered.  An MPC
formulation for mobile manipulators with holonomic bases that allows collision
avoidance was presented in \cite{Avanzini2015}. The perceived obstacles were
translated into a set of spheres to be respected by the MPC scheme. The proposed
approach used dynamically changing weights to change between arm motion and
locomotion, resulting in a locked arm during navigation. A different weight
setting was used to perform motion underneath a horizontal bar with an a priori
position. The work is extended to non\hyp{}holonomic bases and includes object
detection in moving underneath the horizontal bar \cite{Avanzini2018}. 
