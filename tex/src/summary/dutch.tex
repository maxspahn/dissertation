\acresetall
%**Motivatie**  
Omdat robotica een cruciale rol zal spelen in de toekomst van onze moderne
samenlevingen, is de behoefte aan vooruitgang op dit gebied groter dan ooit.
Hoewel robots al aanwezig zijn in industriële omgevingen, zijn ze opvallend
afwezig in dynamische omgevingen. Dynamische omgevingen worden gekenmerkt door
andere bewegende agenten, zoals mensen, variërende taken en hoge
veiligheidseisen. Met als doel robots in dergelijke omgevingen in te zetten,
wordt \ac{tg} cruciaal. \Ac{tg}-benaderingen zijn gericht op het berekenen van
reeksen controlecommando's die de robot van zijn huidige configuratie naar een
gewenste eindtoestand brengen, terwijl botsingen met obstakels en met zichzelf
worden vermeden. Hierdoor bevindt \ac{tg} zich direct tussen taakplanning – het
probleem van welke taken in welke volgorde moeten worden uitgevoerd – en
controle, het probleem van het uitvoeren van motorcommando's. Een goede
oplossing voor \ac{tg} moet snel zijn om veranderingen in de omgeving het hoofd
te bieden, flexibel voor verschillende doeldefinities, en de veiligheid
bevorderen. De meeste vooruitgangen op het gebied van robotica in \ac{tg}
richten zich ofwel op manipulatoren of op mobiele robots. Echter, de combinatie
van beide systemen lijkt onvermijdelijk voor inzet in door mensen gedeelde
omgevingen.

%**\Ac{mpc}-oplossing**  
\Ac{tg} voor mobiele manipulatie wordt meestal geformuleerd als een
optimalisatieprobleem met een eindige tijdshorizon. Deze benadering staat bekend
als \ac{mpc} en wordt veel gebruikt op het gebied van autonoom rijden vanwege de
haalbaarheids- en stabiliteitsgaranties. In \cref{cha:icra_21} presenteren we
een methode om \ac{mpc} toe te passen op mobiele manipulatie. De methode
formuleert het \ac{tg}-probleem voor de gehele kinematische keten en maakt
gebruik van \ac{fsd} voor botsingsvermijding. Dit leidt tot redelijke
controlefrequenties van 10 Hz, onafhankelijk van het aantal obstakels in de
omgeving. Belangrijk is dat deze aanpak een gekoppelde beweging van de mobiele
basis en de manipulator mogelijk maakt. Dit is voordelig in situaties waarin
synchronisatie van de twee subsystemen cruciaal is, zoals bij het openen van
deuren of het verplaatsen van obstakels.

%**Overgang naar \ac{fabrics}**  
Ondanks deze vereenvoudigingen in de omgevingsvoorstellingen, beperken de
rekentijden de toepasbaarheid van \ac{mpc} op mobiele manipulatie, omdat de
beweging niet echt reactief wordt beschouwd en verschillende componenten zoals
doelattractie en botsingsvermijding niet gemakkelijk te scheiden zijn. Een
recent nieuwe benadering voor receding horizon control is de formulering als een
puur geometrisch probleem. Vroege successen in deze richting, waaronder \ac{cic}
en \ac{apf}, leidden tot de formulering als verzamelingen dynamische systemen op
gladde variëteiten in de configuratieruimte. Het kader van \ac{fabrics}
verenigt dergelijke ideeën, biedt stabiliteitsgaranties in statische omgevingen
en resulteert in zeer reactief gedrag vergelijkbaar met eenvoudige laag-niveau
controllers.

%**\Ac{fabrics}-oplossing**  
Dit kader is gebaseerd op niet-Riemanniaanse geometrie om een gladde variëteit
van de configuratieruimte te vormen met individuele gedragingen zoals
botsingsvermijding, gewrichtslimieten en doelattractie. In \cref{cha:tro_23}
presenteren we een generalisatie van \ac{fabrics} naar dynamische omgevingen. We
verwijzen naar het resulterende kader als \ac{df}. De generalisatie maakt
gebruik van tijdgeparameteriseerde variëteiten om bewegende obstakels en
tijdgeparameteriseerde referentietrajecten te integreren. Dit laatste is vooral
belangrijk voor \ac{tg} met lange horizon die lokale minima kan vertonen.
Belangrijk is dat in \cref{cha:tro_23} wordt aangetoond dat de dynamische
component van \ac{df} noodzakelijk is bij het omgaan met bewegende obstakels.
Omdat afstotende krachten in \ac{fabrics} evenredig zijn aan de
naderingssnelheid van obstakel en robot, is een pseudo-statische
botsingsvermijding niet voldoende wanneer de robot langzaam beweegt.

%**Toepassingen in de echte wereld**  
Tot slot implementeren we het algemene kader van \ac{df} in verschillende
real-world omgevingen om de toepasbaarheid van het kader op mobiele manipulatie
te demonstreren. Ten eerste presenteren we een manier om niet-holonome
beperkingen in het kader te integreren. Ondanks het verlies van formele
garanties op convergentie, blijkt de methode de natuurlijke uitbreiding voor
mobiele robots met niet-holonome beperkingen. Ten tweede wordt in
\cref{cha:icra_23} een symbolische implementatie van \ac{fabrics} gepresenteerd
om hogere controlefrequenties te bereiken. Symbolische implementaties zijn
mogelijk omdat het kader van \ac{fabrics} is gebaseerd op
differentiaalvergelijkingen van de tweede orde, waarvoor een gesloten oplossing
bestaat. Voor veranderende omgevingen worden obstakelstatussen dan alleen op
runtime geconcretiseerd. Daarnaast tonen we aan dat symbolische hyperparameters
automatisch kunnen worden afgesteld om een prestatie op expertniveau te
bereiken. Ten derde integreert \cref{cha:ral_24}, om aan de hoge eisen van de
perceptiepijplijn te voldoen, verschillende impliciete omgevingsvoorstellingen
in het kader. Het gebruik van \ac{sdf} en \ac{fsd} is bijvoorbeeld
veelgebruikt in de mobiele robotica bij het formuleren van \ac{tg} als \ac{mpc}.
We laten zien dat dezelfde voorstellingen in \ac{fabrics} kunnen worden
gebruikt, met kortere oplossingsuren als resultaat. Tot slot implementeert
\cref{cha:rss_24} een mobiele manipulator die wordt bestuurd door \ac{fabrics}
in een supermarkt. Behendige manipulaties worden geprogrammeerd met behulp van
leren-uit-demonstratie, met \ac{fabrics} als de onderliggende codering. Dit
maakt het mogelijk om ingewikkelde gedragingen aan te leren in plaats van ze te
programmeren, terwijl de eigenschappen van botsingsvermijding behouden blijven.

%**Samenvatting**  
Deze scriptie presenteert inzichten in aspecten van bewegingsplanning, breidt
het kader van \ac{fabrics} voor \ac{tg} uit en vergelijkt het uitgebreid met de
meer gebruikte methode van \ac{mpc}. Met de ideeën die in deze scriptie worden
gepresenteerd, hopen we het gebruik van geometrische eigenschappen van
robotsystemen in door mensen gedeelde omgevingen te stimuleren. Deze benadering
biedt niet alleen reactieve \ac{tg}, maar kan ook dienen als een compacte
codering van trajecten voor toekomstig gebruik in op leren gebaseerde methoden.
