% summary of below repeated introduction

As robotics will play a crucial role in the future of our
modern societies in the global North, the need for
advancements in the field is more pronounced than ever.
While robots are already present in industrial settings,
they are noticeably absent from human-shared environments.
Human-shared environments are characterized by other moving agents, 
such as humans, varying tasks and high safety requirements.
In the aim to deploy robots in such environments, the
\ac{tg} problem becomes crucial. \Ac{tg} approaches compute sequences
of control commands that bring the robot from its current
configuration to a desired goal state while avoiding
collisions with obstacles and itself. Thus, it is directly placed between 
task planning, the problem of defining what high level tasks should be executed
in which order, and control, the problem of executing motor commands.
A good solution to \ac{tg} must be fast, to cope with changes in the
environment, flexible to different goal definitions, and promote safety.
Most advancements in the field of robotics in \ac{tg} focus either on
manipulators or on mobile robots. However, the combination of both
systems seems inevitable for deployment in human-shared environments.
This dissertation provides an overview of different approaches to \ac{tg} for
mobile manipulation. Then, it presents a method to bring widely used \ac{mpc}
to mobile manipulation. The main part of this thesis focuses on the geometric
framework for \ac{tg} called \ac{fabrics}. In the framework of \ac{fabrics},
individual behaviors are designed as differential equations of second order
for which stability properties are conserved during summation transfer between
different manifolds of the configuration space. This thesis makes several
contributions to the framework of \ac{fabrics} and its application to mobile
manipulation.

\paragraph{\acf{mpc} for mobile manipulation}
The first contribution of this thesis is the introduction of an \ac{mpc}
formulation for mobile manipulation. The method presented in \cref{cha:icra_21}
copes with poor scaling in the number of collision constraints, by using the
concept of \ac{fsd}. This leads to reasonable control frequencies of 10Hz
independent on the amount of collision obstacles in the environment.
Importantly, this approach allowed for coupled motion of the mobile base and the
manipulator. This is beneficial in situations where synchronization of the two
subsystems is crucial, such as opening doors or moving below obstacles. 

\paragraph{Dynamic fabrics}
To achieve truly reactive behavior, \ac{mpc} for whole-body \ac{tg} does not
result in sufficiently fast planning frequencies. A more reactive approach is
known under the name of \ac{fabrics}. This framework relies on non-Riemannian
geometry to shape a smooth manifold of the configuration space with individual
behaviors, such as collision avoidance, joint-limit avoidance, and goal 
attraction. Essentially, it generalizes the ideas of potential field methods
and Cartesian impedance control to a more general setting. The framework comes
with convergence guarantees in static environments, but is still little used
for real applications. In \cref{cha:tro_23}, we present a generalization of
\ac{fabrics} to dynamic environments. We refer to the resulting framework as
\ac{df}. The generalization uses time-parameterized manifolds to integrate 
moving obstacles and time-parameterized reference trajectories. The later is
especially important for long-horizon \ac{tg} that may exhibit local minima.
Importantly, \cref{cha:tro_23} shows that the dynamic component of \ac{df}
is required when coping with moving obstacles, because with slow moving robots,
collision avoidance in a pseudo-static settings is not sufficient.

\paragraph{Fabrics for mobile manipulation}
Finally, this thesis also advances the applicability of \ac{fabrics} to mobile
manipulation by several contributions. 
Next to the introduction of \ac{df}, \cref{cha:tro_23} also presents a way to
integrate non-holonomic constraints into the framework. Despite resulting in a
loss of convergence guarantee, the method is shown to be the natural extension
to 
\Cref{cha:icra_23} presents a symbolic
implementation of \ac{fabrics} to achieve higher control frequencies. The
closed-form solution to the \ac{tg} is parameterized by problem parameters in
case of changing environments. Additionally, we show symbolic 
hyperparameters can be tuned automatically to achieve expert-level performance.
To overcome high requirements on the perception pipeline, \cref{cha:ral_24}
integrates different implicit environment representations into the framework.
Using \ac{sdf} and \ac{fsd} for example is widely used in mobile robotics when
formulating \ac{tg} as \ac{mpc}. We show that the same representations can be
used in \ac{fabrics} while achieving much faster solver times.
\Cref{cha:rss_24} deploys a mobile manipulator controlled by \ac{fabrics}
in a supermarket. Programming dexterous manipulation is programmed using
learning-from-demonstration with \ac{fabrics} as the underlying encoding.
That allowed to teach rather than program complicated behaviors while
maintaining properties on collision avoidance.

% Summary of summary
This thesis presents insights and advances the framework of \ac{fabrics} to
\ac{tg} and compares it extensively to more commonly used methods, such as
\ac{mpc}. Through the ideas presented in this thesis, we hope to encourage the
usage geometric properties of robotic systems deployed to human-shared 
environments. This approach does not only provide reactive \ac{tg} but also may
act as a compact encoding of trajectories for learning-based methods in the
future.










