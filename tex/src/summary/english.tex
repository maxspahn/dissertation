% summary of below repeated introduction

% Motivation
As robotics will play a crucial role in the future of our
modern societies, the need for
advancements in the field is more pronounced than ever.
While robots are already present in industrial settings,
they are noticeably absent from dynamic environments.
Dynamic environments are characterized by other moving agents, 
such as humans, varying tasks and high safety requirements.
With the aim to deploy robots in such environments, \ac{tg} becomes crucial.
\Ac{tg} approaches aim to compute sequences
of control commands that bring the robot from its current
configuration to a desired goal state while avoiding
collisions with obstacles and itself. Thus, it is directly placed between 
task planning, the problem of defining what high level tasks should be executed
in which order, and control, the problem of executing motor commands.
A good solution to \ac{tg} must be fast, to cope with changes in the
environment, flexible to different goal definitions, and it should promote safety.
Most advancements in the field of robotics in \ac{tg} focus either on
manipulators or on mobile robots. However, the combination of both
systems seems inevitable for deployment in human-shared environments.

% Overview
% This dissertation first presents a method to bring widely used \ac{mpc} to
% mobile manipulation. Then, the main part of this thesis focuses on the geometric
% framework for \ac{tg} called \ac{fabrics}. In the framework of \ac{fabrics},
% individual behaviors are designed as differential equations of second order for
% which stability properties are conserved during summation and transfer between
% different manifolds of the configuration space. This thesis makes several
% contributions to the framework of \ac{fabrics} and its application to mobile
% manipulation.

% MPC solution
\Ac{tg} for mobile manipulation is usually formulated as an
optimization problem of a finite time horizon. This approach is known as
\ac{mpc} and is widely used in the field of autonomous driving thanks to its
feasibility and stability guarantees. In \cref{cha:icra_21}, we present a method
to bring \ac{mpc} to mobile manipulation. The method formulates the \ac{tg}
problem for the entire kinematic chain and relies on \ac{fsd} for collision
avoidance. This leads to reasonable control frequencies of 10Hz independent on
the amount of collision obstacles in the environment. Importantly, this approach
allows for coupled motion of the mobile base and the manipulator. This is
beneficial in situations where synchronization of the two subsystems is crucial,
such as opening doors or moving obstacles around.
% Transition to fabrics
Despite these simplications on the environment representations, computational costs
limit the applicability of \ac{mpc} to mobile manipulation as motion is not
considered truly reactive and different components, such as goal attraction and
collision avoidance, are not easily separable.

A recent novel approach to
receding horizon control is the formulation as a purely geometric problem.
Early successes in this direction, including \ac{cic} and \ac{apf}, led to the
formulation as sets of dynamical systems on smooth manifolds in the
configuration space. The framework of \acf{fabrics} unifies such ideas, 
offers stability guarantees in static environments, and results in highly
reactive behavior similar to simple low-level controllers.
% Fabrics solution
This framework relies on non-Riemannian
geometry to shape a smooth manifold of the configuration space with individual
behaviors, such as collision avoidance, joint-limit avoidance, and goal 
attraction. In \cref{cha:tro_23}, we present a generalization of
\ac{fabrics} to dynamic environments. We refer to the resulting framework as
\ac{df}. The generalization uses time\hyp{}parameterized manifolds to integrate 
moving obstacles and time\hyp{}parameterized reference trajectories. The latter is
especially important for long-horizon \ac{tg} that may exhibit local minima.
Importantly, \cref{cha:tro_23} shows that the dynamic component of \ac{df}
is required when coping with moving obstacles. As repulsive forces 
in \ac{fabrics} are proportional to the approaching speed of obstacle and robot,
collision avoidance in a pseudo-static fashion is not sufficient when the robot
is moving slowly.
% Real world settings
Finally, we deploy the general framework of \ac{df} to several real-world
settings showing the applicability of the framework to mobile manipulation.
First, we present a way to
integrate non\hyp{}holonomic constraints into the framework. Despite loosing
formal guarantees on convergence, the method is shown to be the natural extension
to wheeled mobile robots characterized by non\hyp{}holonomic constraints.
Second, \cref{cha:icra_23} presents a symbolic
implementation of \ac{fabrics} to achieve higher control frequencies.
Symbolic implementations are possible because the framework of \ac{fabrics}
is based on differential equations of second order, for which a closed-form
solution exists. For changing environments, obstacles states are then only
concretized at runtime.
Additionally, we show symbolic 
hyperparameters can be tuned automatically to achieve expert-level tuning performance.
Third, to overcome high requirements on the perception pipeline, \cref{cha:ral_24}
integrates different implicit environment representations into the framework.
Using \acf{sdf} and \acf{fsd} for example is widely used in mobile robotics when
formulating \ac{tg} as \ac{mpc}. We show that the same representations can be
used in \ac{fabrics} while achieving faster solver times.
Finally, 
\cref{cha:rss_24} deploys a mobile manipulator controlled by \ac{fabrics}
in a supermarket. Dexterous manipulation is programmed using
learning-from-demonstration, with \ac{fabrics} as the underlying encoding.
That allows to teach rather than program complicated behaviors while
maintaining properties on collision avoidance.
% Summary of summary

This thesis presents insights into aspects of motion planning, advances the
framework of \ac{fabrics} for \ac{tg}, and compares it extensively to the more
commonly used method of \ac{mpc}. Through the ideas presented in this thesis, we
hope to encourage the usage of geometric properties of robotic systems deployed
to human-shared environments. This approach does not only provide reactive
\ac{tg} but also may act as a compact encoding of trajectories for
learning-based methods in the future.










