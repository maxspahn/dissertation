\acresetall
%**Motivation**  
Da Robotik eine zentrale Rolle in der Zukunft unserer modernen Gesellschaften
spielen wird, ist der Bedarf an Fortschritten in diesem Bereich ausgeprägter
denn je. Während Roboter bereits in industriellen Umgebungen präsent sind,
fehlen sie auffallend in dynamischen Umgebungen. Dynamische Umgebungen sind
durch andere bewegliche Akteure wie Menschen, wechselnde Aufgaben und hohe
Sicherheitsanforderungen gekennzeichnet. Um Roboter in solchen Umgebungen
einzusetzen, wird die \ac{tg} entscheidend. \Ac{tg}-Ansätze zielen darauf ab,
Sequenzen von Steuerbefehlen zu berechnen, die den Roboter von seiner aktuellen
Konfiguration zu einem gewünschten Zielzustand bringen und dabei Kollisionen mit
Hindernissen und sich selbst vermeiden. Somit ist \ac{tg} direkt zwischen der
Aufgabenplanung – dem Problem, welche Aufgaben in welcher Reihenfolge ausgeführt
werden sollen – und der Steuerung – dem Problem der Ausführung von Motorbefehlen
– angesiedelt. Eine gute Lösung für \ac{tg} muss schnell sein, um auf Änderungen
in der Umgebung reagieren zu können, flexibel für verschiedene Zieldefinitionen
und sicherheitsfördernd. Die meisten Fortschritte im Bereich der Robotik in
Bezug auf \ac{tg} konzentrieren sich entweder auf Manipulatoren oder auf mobile
Roboter. Die Kombination beider Systeme scheint jedoch für den Einsatz in von
Menschen geteilten Umgebungen unvermeidlich.

%**\Ac{mpc}-Lösung**  
\Ac{tg} für mobile Manipulation wird häufig als Optimierungsproblem über
einen endlichen Zeithorizont formuliert. Dieser Ansatz ist als \ac{mpc} bekannt
und wird im Rahmen von autonomen Fahren aufgrund seiner Machbarkeits- und
Stabilitätsgarantien weit verbreitet eingesetzt. In \cref{cha:icra_21}
präsentieren wir eine Methode, um \ac{mpc} auf mobile Manipulation anzuwenden.
Die Methode formuliert das \ac{tg}-Problem für die gesamte kinematische Kette
und nutzt \ac{fsd} zur Kollisionsvermeidung. Dies führt zu ausreichend schnellen
Steuerfrequenzen von 10 Hz, unabhängig von der Anzahl der Kollisionshindernisse
in der Umgebung. Wichtig ist, dass dieser Ansatz eine gekoppelte Bewegung der
mobilen Basis und des Manipulators ermöglicht. Dies ist vorteilhaft in
Situationen, in denen die Synchronisation der beiden Teilsysteme entscheidend
ist, wie beim Öffnen von Türen oder beim Bewegen von Hindernissen.

%**Übergang zu \ac{fabrics}**  
Trotz dieser Vereinfachungen in der Umgebungsdarstellung schränken die
Rechenkosten die Anwendbarkeit von \ac{mpc} auf die mobile Manipulation ein, da
die Bewegung nicht wirklich reaktiv betrachtet wird und verschiedene Komponenten
wie Zielanziehung und Kollisionsvermeidung nicht leicht trennbar sind. Ein neuer
Ansatz zu Receding-Horizon-Control ist die Formulierung als rein
geometrisches Problem. Erste Erfolge in diese Richtung, einschließlich \ac{cic}
und \ac{apf}, führten zur Formulierung als Menge von dynamischen Systemen auf
glatten Mannigfaltigkeiten im Konfigurationsraum. Der Rahmen von \ac{fabrics}
vereinheitlicht solche Ideen, bietet Stabilitätsgarantien in statischen
Umgebungen und ermöglicht ein hochreaktives Verhalten ähnlich wie bei einfachen
niederen Steuerungen.

%**\Ac{fabrics}-Lösung**  
Dieser Rahmen basiert auf nicht-riemannscher Geometrie, um eine glatte
Mannigfaltigkeit des Konfigurationsraums mit individuellen Verhaltensweisen wie
Kollisionsvermeidung, Gelenkgrenzenvermeidung und Zielanziehung zu gestalten. In
\cref{cha:tro_23} präsentieren wir eine Verallgemeinerung von \ac{fabrics} auf
dynamische Umgebungen. Wir bezeichnen den resultierenden Rahmen als \ac{df}. Die
Verallgemeinerung nutzt zeitparameterisierte Mannigfaltigkeiten zur Integration
von sich bewegenden Hindernissen und zeitparameterisierten Referenztrajektorien.
Letzteres ist besonders wichtig für langhorizontale \ac{tg}, die lokale Minima
aufweisen können. Wichtig ist, dass in \cref{cha:tro_23} gezeigt wird, dass die
dynamische Komponente von \ac{df} notwendig ist, um mit beweglichen Hindernissen
umzugehen. Da abstoßende Kräfte in \ac{fabrics} proportional zur
Annäherungsgeschwindigkeit von Hindernis und Roboter sind, ist eine
pseudo-statische Kollisionsvermeidung nicht ausreichend, wenn sich der Roboter
langsam bewegt.

%**Einstellungen in der realen Welt**  
Schließlich setzen wir das allgemeine Rahmenwerk von \ac{df} in mehreren realen
Umgebungen ein, um die Anwendbarkeit des Rahmens auf die mobile Manipulation zu
demonstrieren. Zuerst präsentieren wir eine Methode, um nicht-holonome
Einschränkungen in das Rahmenwerk zu integrieren. Trotz des Verlusts formaler
Konvergenzgarantien erweist sich die Methode als natürliche Erweiterung auf
mobile Roboter mit nicht-holonomischen Einschränkungen. Zweitens wird in
\cref{cha:icra_23} eine symbolische Implementierung von \ac{fabrics}
vorgestellt, um höhere Steuerfrequenzen zu erreichen. Symbolische
Implementierungen sind möglich, da der Rahmen von \ac{fabrics} auf
Differentialgleichungen zweiter Ordnung basiert, für die eine geschlossene
Lösung existiert. Für sich ändernde Umgebungen werden die Hinderniszustände dann
nur zur Laufzeit konkretisiert. Außerdem zeigen wir, dass symbolische
Hyperparameter automatisch abgestimmt werden können, um eine
Experten-Tuning-Leistung zu erreichen. Drittens integriert \cref{cha:ral_24}, um
hohe Anforderungen an die Wahrnehmungspipeline zu überwinden, verschiedene
implizite Umgebungsdarstellungen in den Rahmen. Die Verwendung von \ac{sdf} und
\ac{fsd} wird beispielsweise häufig in der mobilen Robotik verwendet, wenn
\ac{tg} als \ac{mpc} formuliert wird. Wir zeigen, dass dieselben Darstellungen
in \ac{fabrics} verwendet werden können und dabei schnellere Lösungszeiten
erreicht werden. Schließlich wird in \cref{cha:rss_24} ein mobiler Manipulator,
der von \ac{fabrics} gesteuert wird, in einem Supermarkt eingesetzt. Geschickte
Manipulationen werden mittels Lernen aus Demonstrationen programmiert, wobei
\ac{fabrics} als zugrunde liegende Kodierung verwendet wird. Dadurch wird es
möglich, komplizierte Verhaltensweisen zu lehren anstatt sie zu programmieren,
während die Eigenschaften der Kollisionsvermeidung erhalten bleiben.

%Zusammenfassung
Diese Dissertation präsentiert Einblicke in verschiedene Aspekte der
Bewegungsplanung, erweitert den Rahmen von \ac{fabrics} für die \ac{tg} und
vergleicht ihn umfassend mit der häufiger verwendeten Methode der \ac{mpc}. Mit
den in dieser Dissertation vorgestellten Ideen hoffen wir, die Nutzung
geometrischer Eigenschaften von Robotersystemen in von Menschen geteilten
Umgebungen zu fördern. Dieser Ansatz bietet nicht nur eine reaktive \ac{tg},
sondern kann auch als kompakte Kodierung von Trajektorien für zukünftige
lernbasierte Methoden dienen.
