\paragraph{Implicit Environment Represenations for trajectory generation}

\begin{figure}
  \centering
  \input{src/24-spahn-ral/img/methods/inkscape/spectrum_tex.pdf_tex}
  \caption{Different levels of implicitness for environment representations.}
  \label{fig:ral24_overview}
\end{figure}

While \ac{fabrics} mainly employ explicit environment
representations \cite{Spahn2023,Ratliff2020}, new \ac{tg}
methods lean towards implicit approaches.
For example, representing the environment's free space as 
a set of half-planes has proven sucessful for whole-body
\ac{mpc} formulations \cite{Spahn2021}.
%For \ac{tg} using
%short-term optimization, unoccupied space constraints limit
%robot movement, proven useful for mobile manipulators in
%cluttered areas \cite{Spahn2021}.
In drone flight, a similar
concept generates safe flight zones along a global path
\cite{Liu2017a,Tordesillas2019a}. In the context of drone
flying, \ac{sdf} has been utilized with \ac{mpc} in unknown
environments \cite{Oleynikova2017voxblox}. Raw lidar data
has been used in combination with \acp{rmp} showing
impressively high frequencies when computation is parallized
on GPU \cite{Pantic2023obstacle}. Recent advances in
sampling-based \ac{mpc}, also referred to as \ac{mppi}, utilizing physics engines for
collision avoidance \cite{Pezzato2023sampling}, integrate
obstacle collisions in cost functions during trajectory
rollouts. A similar approach is seen in
\cite{Sundaralingam2023curobo}.

% All previous works focus on implicit environment
% representations. However, most of these methods still
% require manual robot representation, like composing spheres.
% On that end, some exploration of more implicit
% characterizations, such as learned \ac{sdf} like in
% \cite{Liu2022regularized,Koptev2023neural}, is underway.
