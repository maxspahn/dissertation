\section{Introduction}
\label{sec:intro}

As robots are making their way into human shared environments,
fast reactive behavior is needed to make sure that obstacles are safely avoided
at all time. Trajectory optimization methods, such as Model Predictive Control,
are widely used to guarantee collision avoidance during execution. While such
methods perform well in slowly changing environments, their computational costs
limit the achievable computation frequencies cannot be considered truly reactive. 
\ac{fabrics} offer an
alternative to classic trajectory optimization techniques. Based on differential
geometry, policies are composed of several components to form a highly reactive
and fast behavior. However, the composition of \ac{fabrics} of
individual obstacle avoidance geometries is limited to simple geometric shapes,
such that a differentiable distance function can easily be formulated. The 
sets a challenging requirement on the perception part of the motion generation
pipeline. In this work, we present and analyse three different methods to 
overcome this drawback, namely integration of \ac{fsd}, 
\acp{sdf} and raw sensor data into the framework of
\ac{fabrics}. While putting more computational effort onto the fabrics planner, it
relaxes the computational costs of the perception pipeline.
In the process, we derive essential extensions to the framework and
analyse strength and weaknesses of the individual methods. To summarize, this
paper makes the following contributions:
\begin{itemize}
  \item We integrate implicit representation of the environment into the
    framework of \ac{fabrics}, namely \ac{fsd}, \ac{sdf}
    and raw sensor data.
  \item We derive how numerical gradients can be used for pullback operations
    which are essential in the composition of \ac{fabrics}.
  \item We analyse strength and weaknesses of the three representations in
    dynamic environments.
  \item We present real-world experiments illustrating the power of our
    open-source implementation.
\end{itemize}

In this project, direct sensor data should be integrated into
the framework of \ac{fabrics} to remove the requirement for formal
object detection algorithms. Collision avoidance is crucial for every robotic
application in a human-shared environment. Model Predictive Control is highly
valuable to mobile robots as it leverages formal safety guarantees to mobile
robots. However, the underlying optimization problem may be highly non-linear
and difficult to solve, such that control frequencies higher than 10Hz are generally
hard to achieve \cite{hewing2020learning}. For mobile manipulators, it
becomes even more difficult to obtain reasonable control-frequencies. On the
other hand, \ac{fabrics} offer a highly reactive motion planning
framework for various types of robots \cite{Ratliff2021}.
Frequencies above 1kHz are realistic to obtain. However, so far, \ac{fabrics}
depend on precise perception tools to realize safe motions.
Specifically, \ac{fabrics} require a differentiable distance function
between robot and obstacle. Sensors, such as Lidar or depth cameras, might
directly offer such distance measures. 

\begin{figure}
  \centering
  \input{src/24-spahn-ral/img/methods/inkscape/spectrum_tex.pdf_tex}
  \caption{Different levels of implicitness for environment representations.}
  \label{fig:overview}
\end{figure}


