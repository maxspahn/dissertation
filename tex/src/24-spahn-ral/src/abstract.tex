\begin{abstract}
Deployment of robots in dynamic environments requires reactive trajectory
generation. In contrast to optimization-based methods, such as \acl{mpc},
\acl{fabrics} offer a computationally efficient way to generate trajectories
that include all avoidance behaviors if the environment can be represented as a
set of object primitives. Obtaining such a representation from sensor data is
challenging, especially in dynamic environments. In this paper, we integrate
\textit{implicit} environment representations, such as \aclp{sdf} and \acl{fsd}
into the framework of \acl{fabrics}. In the process, we derive how numerical
gradients can be integrated into the push and pull operations in \acl{fabrics}.
Our experiments reveal that both, ground robots and robotic manipulators, can be
controlled using these implicit representations. Moreover, we show that, unlike
the explicit representation, implicit representations can be used in the
presence of dynamic obstacles without further considerations. Finally, we
demonstrate our methods in the real-world, showing the applicability of our
approach in practice.
\end{abstract}

% for the submission
%Deployment of robots in dynamic environments requires reactive trajectory generation. In contrast to optimization- based methods, such as Model Predictive Control, Geometric Fabrics offer a computationally efficient way to generate trajectories that include all avoidance behaviors if the environment can be represented as a set of object primitives. Obtaining such a representation from sensor data is challenging, especially in dynamic environments. In this paper, we integrate implicit environment representations, such as Signed Distance Fields and Free Space Decomposition into the framework of Geometric Fabrics. In the process, we derive how numerical gradients can be integrated into the push and pull operations in Geometric Fabrics. Our experiments reveal that both, ground robots and robotic manipulators, can be controlled using these implicit representations. Moreover, we show that, unlike the explicit representation, implicit representations can be used in the presence of dynamic obstacles without further considerations. Finally, we demonstrate our methods in the real-world, showing the applicability of our approach in practice.

