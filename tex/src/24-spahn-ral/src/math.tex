\section{Background}
\label{sec:math}
%
We introduce some notations and the mathematical operations 
from differential geometry used in the framework of
\ac{fabrics}. This introduction is rather limited and the
reader is referred to \cite{Ratliff2021,Spahn2023,Wyk2022} for a more in-depth
introduction.
%
\subsection{Configurations and task variables}%
\label{sub:configurations_and_task_variables}
%
We denote $\q\in\Q\subset\Rn$ a configuration of the robot with $n$ its degrees
of freedom; \Q{} is the configuration space of the generalized coordinates of
the system. Generally, $\q(t)$ defines the robot's configuration at time $t$, so
that \qdot, \qddot{} define the instantaneous derivatives of the robot's
configuration. Similarly, we assume that there is a set of task variables
$\xj\in\Xj\subset\Rmj$ with variable dimension $m_j \leq n$. The task space
\Xj{} defines an arbitrary manifold of the configuration space \Q{} in which a
robotic task can be represented. Further, we assume that there is a differential
map $\map_j:\Rn\rightarrow\Rmj$ that relates the configuration space to the
$j^{th}$ task space. For example, when a task variable is defined as the
end-effector position, then $\map_j$ is the positional part of the forward
kinematics. On the other hand, if a task variable is defined to be the joint
position, then $\map_j$ is the identity function. In the following, we drop the
subscript $j$ in most cases for readability when the context is clear.

We assume that \map{} is in $\mathcal{C}^1$ so that the Jacobian is
defined as
\begin{equation}
  \J = \derf{\q}{\map} \in \mathcal{R}^{m\times n}, 
\end{equation}
or $\J = \der{\q}{\map}$ for short.
Thus, we can write the total time derivatives of \x{} as
$\xdot = \J\qdot$ and $\xddot = \J\qddot + \Jdot\qdot$.
%
\subsection{Spectral semi-sprays}%
\label{sub:semi_spectral_sprays}
%
The framework of
\ac{fabrics} designs trajectory generation as second-order dynamical systems $\xddot =
\pi(\x,\xdot)$~\cite{Cheng2020,Ratliff2020}. The trajectory is defined by the
differential equation $\M\xddot + \f = 0$, where $\M(\x,\xdot)$ and
$\f(\x,\xdot)$ are functions of position and velocity. Besides, \M{} is
symmetric and invertible. We denote such systems as $\Spec = \spec$ and refer to
them as \textit{spectral semi-sprays}, or \textit{specs} for
short. Often, we drop the subscript $\X$ when the context is clear.
%
\subsection{Operations on specs}%
\label{sub:operations_on_specs}
%
Trajectory generation requires the combination of multiple
components, such as collision avoidance, joint limits
avoidance, etc.
In the framework of \ac{fabrics}, these components are
represented as specs in different manifolds and combined in a consistent way using
a metric-weighted sum.
Related operations from differential geometry are recalled
here.

Given a differential map $\map: \Q\rightarrow\X$ and a spec \spec{}, the \textit{pullback}
is defined as 
\begin{equation}
  \pull{\map}{\spec} = {\left(\Jt\M\J, \Jt(\f+\Jdot\qdot)\right)}_{Q}.
  \label{eq:pullback}
\end{equation}
The pullback allows converting between two distinct manifolds (e.g. a spec could be 
defined in the robot's workspace and pulled into the robot's configuration space using
the pullback with \map{} being the forward kinematics).

For two specs, $\Spec_1 = {\left(\M_1,\f_1\right)}_{\X}$ and 
$\Spec_2 = {\left(\M_2,\f_2\right)}_{\X}$, their \textit{summation} is defined by:
\begin{equation}
  \Spec_1 + \Spec_2 = {\left(\M_1 + \M_2, \f_1 + \f_2\right)}_{\X}.
  \label{eq:specs_summation}
\end{equation}
%

Additionally, a spec can be \textit{energized} by a Lagrangian energy. Effectively, 
this equips the spec with a metric.
Specifically, given a spec of form $\Spec_{\vec{h}} = (\mat{I},\vec{h})$ and 
an energy Lagrangian \le{} with the derived equations of motion $\M_{\le}\xddot + \f_{\le} =0$, 
we can define the operation
\begin{equation}
  \begin{split}
  S_{\vec{h}}^{\le} &= \text{energize}_{\le}\{S_{\vec{h}}\} \\
    &= (\Me, \fe + \Pe[\Me\vec{h} - \fe]), 
  \end{split}
  \label{eq:energization}
\end{equation}
where $\Pe = \Me\left(\Me^{-1} - \frac{\xdot\xdot^T}{\xdot^T\Me\xdot}\right)$ is an
orthogonal projector. The resulting spec is an \textit{energized spec} and 
we call the operation \textit{energization}.

With spectral semi-sprays and the presented operations,
avoidance behavior, such as joint limit avoidance, collision
avoidance or self-collision avoidance, can be realized.

\subsection{Optimization fabrics}%
\label{sub:optimization_fabrics}
%
In the previous subsection, we recalled how different avoidance behaviors can
be combined. Spectral semi-sprays can additionally be \textit{forced} by a
potential, denoted as the \textit{forced variant} of form $\Spec_{\forc} =
\left(\M,\f + \der{\x}{\forc}\right)$. This forcing term clearly changes the
behavior of the system. Under certain conditions, it can be
shown that the trajectory $\x(t)$ of the forced variant converges towards the minimum of the potential \forc{}~\cite{Ratliff2020}.

First, the initial spec that represents an avoidance
component is written in the form $\xddot + \vec{h}(\x,\xdot)
= 0$, such that $\vec{h}$ is \textit{homogeneous of degree
2}: $\vec{h}(\x,\alpha\xdot) = \alpha^2\vec{h}(\x,\xdot)$
(\textbf{Creation}). Secondly, the geometry is energized
(\cref{eq:energization}) with a Finsler
structure~\cite[Definition 5.4]{Ratliff2020}
(\textbf{Energization}). The property of homogeneity of
degree 2 and the energization with the Finsler structure
guarantees, according to ~\cite[Theorem 4.29]{Ratliff2020},
that the energized spec forms a \textit{frictionless
fabric}. A frictionless fabric is defined to optimize the
forcing potential \forc{} when being damped by a positive
definite damping term~\cite[Definition 4.4]{Ratliff2020}.
Thirdly, all avoidance components are combined in the
configuration space of the robot using the pullback and
summation operation (\textbf{Combination}). Note, that both
operations are closed under the algebra designed by these
operations, i.e. every pulled optimization fabric or the sum
of two \ac{fabrics} is, itself, an optimization fabric. In
the last step, the combined spec is forced by the potential
\forc{} with the desired minimum and damped with a positive
definite damping term (\textbf{Forcing}). This resulting
system of form $\M\qddot + \f(\q, \qdot) + \der{\q}{\forc} +
\beta\qdot = 0$ is solved to obtain the trajectory
generation policy in acceleration form $\qddot = \pi(\q,\qdot)$.

