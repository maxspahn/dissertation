\section{Related works}
\label{sec:state}

Initially, we provide an overview of recent developments in geometric control
applied to trajectory generation. Subsequently, we encapsulate implicit
environment representations employed within alternative trajectory generation
approaches.
%
\subsection{Geometric control for trajectory generation}
%
Operational space control emerged as a pioneering control approach, entailing
the imposition of a desired dynamical system onto robotic systems
\cite{Khatib1985}. The foundation of this
concept was generalized under the name of geometric control, where the
exploration of differential geometry offers stability and convergence
through geometric prerequisites \cite{bullo2019geometric}.

More recently, \ac{rmp} for manipulation tasks have
introduced a highly responsive trajectory generation methodology
\cite{Ratliff2018,Cheng2020}. This approach attains composability through a
separation between the importance metric and the forcing
term.
Leveraging the \textit{pullback} and \textit{pushforward} operators to navigate
across configuration space manifolds, distinct behaviors such as collision
avoidance and goal attraction can be systematically devised. Nevertheless, the
design of \acp{rmp} necessitates an intuitive understanding
and experiential experience,
with convergence being conditionally established \cite{Ratliff2020}.

Subsequently, \ac{fabrics} emerged as a novel approach,
entirely seperating the importance metrics and the defining
geometry. By adhering to simple construction principles
governing these two components, convergence can be readily
verified \cite{Ratliff2020,Xie2021,Li2021,meng2019neural}.
In our preceding work on \ac{fabrics}, this framework was
initially applied to mobile manipulation and subsequently
extended to encompass more dynamic environments
\cite{Spahn2023}.

\subsection{Implicit Environment Represenations for trajectory generation}

While \ac{fabrics} mainly focus on explicit environment
representation \cite{Spahn2023,Ratliff2020}, new \ac{tg}
methods lean towards implicit approaches. For \ac{tg} using
short-term optimization, unoccupied space constraints limit
robot movement, proven useful for mobile manipulators in
cluttered areas \cite{Spahn2021}. In drone flight, a similar
concept generates safe flight zones along a global path
\cite{Liu2017a,Tordesillas2019a}. In the context of drone
flying, \ac{sdf} has been utilized with \ac{mpc} in unknown
environments \cite{Oleynikova2017voxblox}. Raw lidar data
has been used in combination with \acp{rmp} showing
impressively high frequencies when computation is parallized
on GPU \cite{Pantic2023obstacle}. Recent advances in
sampling-based \ac{mpc}, utilizing physics engines for
collision avoidance \cite{Pezzato2023sampling}, integrate
obstacle collisions in cost functions during trajectory
rollouts. A similar approach is seen in
\cite{Sundaralingam2023curobo}.

All previous works focus on implicit environment
representations. However, most of these methods still
require manual robot representation, like composing spheres.
On that end, some exploration of more implicit
characterizations, such as learned \ac{sdf} like in
\cite{Liu2022regularized,Koptev2023neural}, is underway.
