\section{Related works}
\label{sec:state}

Initially, we provide an overview of recent developments in geometric control
applied to trajectory generation. Subsequently, we encapsulate implicit
environment representations employed within alternative trajectory generation
methodologies.

\subsection{Geometric control for trajectory generation}
%
Operational space control emerged as a pioneering control approach, entailing
the imposition of a desired dynamical system onto robotic systems
\cite{Khatib1985,Khatib1987}. This notion marked a pivotal stride towards the
innate manipulation of kinematically redundant robots. The foundation of this
concept was solidified within the realm of geometric control, where the
exploration of differential geometry engendered stability and convergence
through geometric prerequisites \cite{bullo2019geometric}.

More recently, \acp{rmp} for manipulation tasks have
introduced a highly responsive trajectory generation methodology
\cite{Ratliff2018,Cheng2019}. This methodology attains composability through a
segregation between the significance metric and the compelling factor.
Leveraging the \textit{pullback} and \textit{pushforward} operators to navigate
across configuration space manifolds, distinct constituents such as collision
avoidance and goal attraction can be systematically devised. Nevertheless, the
design of \acp{rmp} necessitates an intuitive understanding and experiential finesse,
with convergence being conditionally established \cite{Ratliff2020}.

Subsequently, \ac{fabrics} emerged as a novel approach, entirely
disentangling the significance metrics and the defining geometry. By adhering to
straightforward construction principles governing these two components, assured
convergence can be readily ascertained
\cite{Ratliff2020,Xie2021,Li2021,meng2019neural}. In our preceding work on
\ac{fabrics}, this framework was initially
applied to mobile manipulation and subsequently extended to encompass more
dynamic environments \cite{Spahn2022}.

\subsection{Implicit Environment Represenations for trajectory generation}

While \ac{fabrics} mainly focus on explicit environment representation
\cite{Spahn2022,Ratliff2020}, new trajectory methods lean towards implicit
approaches. For trajectory planning using short-term optimization, unoccupied
space constraints limit robot movement, proven useful for mobile manipulators in
cluttered areas \cite{Spahn2021}.
In drone flight, a similar concept generates safe flight zones along a global
path \cite{Liu2017a,Tordesillas2019a}. In the context of drone flying,
\ac{sdf} has been utilized with \ac{mpc} trajectory generation in unknow
environments \cite{Oleynikova2017voxblox}. Raw lidar data has been used in combination
with \acp{rmp} showing impressively high frequencies when computation is
parallized on GPU \cite{Pantic2023obstacle}.
Recent advances in sampling-based
\ac{mpc}, utilizing physics engines for collision avoidance
\cite{Pezzato2023sampling},
integrate obstacle collisions in cost functions during trajectory rollouts. A
parallel approach is seen in \cite{Sundaralingam2023curobo}.

Though many methods demand manual robot representation, like composing spheres,
exploration of more implicit characterizations, such as learned \ac{sdf} like in
\cite{Liu2022regularized,Koptev2023neural}, is underway.
