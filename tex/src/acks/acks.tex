\chapter*{Acknowledgments}
%\addcontentsline{toc}{chapter}{Acknowledgements}
% Set header content
\markboth{Acknowledgments}{Acknowledgments}
\label{acknowledgements}

Pursuing a Ph.D. requires being very privileged in a privileged society and
being surrounded by amazing people. I am grateful for having been in such a
position for the past four years. This time has taught me how engineering
sciences work and has shaped me as a person in ways I could not have imagined.

First and foremost, I would like to thank my supervisors, Javier Alonso-Mora and
Martijn Wisse, for their guidance, support, and patience with me. You, Javier,
introduced me to the world of science and guided me through it. You were able to
endure my stubbornness and my complaints, explained to me the importance of
rigor, and showed me how to be a good scientist.
Towards the end of my
journey, and still today, I am grateful for all the exposure to other research
groups you gave me. Without you, Martijn, my journey might have stopped
much earlier, because the academic world is competitive and often an unhealthy
environment. Your interpretation of the role as a professor kept me going and
often times reminded me why I started the Ph.D. in the first place. It is
refreshing and inspiring how you combine low-level issue-solving, such as robot
repairs, with high-level political communication.

I would like to thank the members of my committee for their time and effort in
reviewing my work. I am grateful for the feedback and the discussions we had.

Being part of the Cognitive Robotics department was a great opportunity,
because I was surrounded by inspiring people and was allowed to shape this
environment. I would like to especially thank Corrado, because he was always one
step ahead of me to show me around without the slightest trace of arrogance.
Making these demos possible would not have been possible without you. Chadi,
the most stable pole in this hectic world, thank you for always staying calm
around me and sharing the interest in robotics software and vim. Most of the
work in this thesis would not have been possible without you. Being a very
opinionated person myself, I could not be more thankful for having met you,
Giovanni. You slowly convinced me of the concept of learning-from-demonstration
and never hesitated to start a heated discussion on the topic of robotic
manipulation, or anything else for that matter. The most thrilling and exciting
moments in this journey were because of you, \#platonics. Mariano, thank you for
keeping a calm mind when around Giovanni and me, both when trying to figure out
rotations and when deciding on where to have lunch. I also want to thank Bruno,
for hinting me towards the geometric approaches, when I was looking for a
meaningful research direction. Spending time in the office or the lab would
have been very a lonely experience without my fellows. I enjoyed the company of
Saray, I expect you to maintain the fabrics package, Luzia for the good
anecdotes from the Länd, Elia, Alvaro, Yuije, Rodrigo, Andreu, Tomas.
I also want to thank the students I had the pleasure to supervise. Evelien and
Paul, thank you for the hard work on path planning and affordances. It was fun
working with you, especially at times, when we were all isolated at home. Thank you, Alex, for accepting the challenge of exploring learning-based methods with
me. Thank you, Caspar, for helping to shape the idea of implicit environment
representations for \ac{fabrics}. I would also like to thank all the
secretaries and the technical staff at CoR for their support and help with us, 
the often stubborn Ph.D. students.

Maxi, thank you for teaming up in this journey and for all the healthy
distractions from the often times, toxic academic environment. I always
enjoyed meeting up for dinner and board games with you and Toni. Thank you,
Patrick, for all the phone calls discussing natural language processing and
startup-life and your persistent demanding for explanations on why I stayed
away from machine learning approaches in my research. I am grateful to having
met you, Emily, Max and Rose, you do not only were my companions in this
exciting journey of parenthood, but also the reason for the most pleasant years
in the Netherlands.

If I am here defending my Ph.D. thesis, it is only because of you, Andrea and
Roland. Throughout my life, you offered my all the opportunities that allowed
me to be here now. You gave me advice when I needed it and gave me the freedom
to explore the world, what I now understand is the hardest for parents. I am so
thankful for all the support you offered me without expecting much in return in
times. Doing a Ph.D. was often mentally demanding, and I would have not managed
it without you, Merle. You helped me through the tough moments, that hardly
anyone was aware of. You organized my Ph.D. more than I did, you reminded me of
deadlines and courses we had to take, but most importantly, you gave me joy in
Delft, shared all experiences and teamed up with me for this journey. During my
Ph.D., I also enjoyed your birth, Luna. The most joyful moment I could think
of. Through the years, you showed me how unimportant my job is in comparison
and how incapable the robots are we develop. It is truly heartwarming to have
you in my life, waiting for me to come back in the evening and wanting to run
around the apartment giggling. My girls, we went through this together and I
do not want to have missed it or experienced it without you. 






