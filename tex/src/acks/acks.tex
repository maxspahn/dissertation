\chapter*{Acknowledgements}
%\addcontentsline{toc}{chapter}{Acknowledgements}
\label{acknowledgements}

% Notes: Javier, Martijn, Comittee, Collegues (Corrado, Maxi, Giovanni, Chadi,
% etc.)
% Notes: family, friends
% Notes: Merle and Luna

% Thanking supervisory teams
% tha

Pursuing a Ph.D. requires to be very previliged in a previliged society and
being surrounded by amazing people. I am grateful for having been in such a
position for the past four years. This time has taught me how engineering
sciences work and has shaped me as a person in ways I could not have imagined.

First and foremost, I would like to thank my supervisors, Javier Alonso-Mora and
Martijn Wisse, for their guidance, support, and patience with me. You, Javier,
introduced me to the world of science and guided me through it. You were able to
endure my stubbornness and my complaints, explained to me the importance of
rigor, and showed me how to be a good scientist.
Towards the end of my
journey, and still today, I am grateful for all the exposure to other research
groups you gave me. Without you, Martijn, my journey would might have stopped
much earlier, because the academic world is competetive and often an unhealthy
environment. Your intepretation of the role as a professor kept me going and
often times reminded me why I started the Ph.D. in the first place. It is
refreshing and inspiring how you combine low-level issue-solving, such as robot
repairs, with high-level political communication.

I would like to thank the members of my committee for their time and effort in
reviewing my work. I am grateful for the feedback and the discussions we had.

Being part of the Cognitive Robotics department was a great opportunity, because
I was surrounded by amazing people. I would like to especially thank Corrado,
because he was always one step ahead of me to show me around without the
slightest trace of arrogance. Making these demos possible would not have been
possible without you. Chadi, the most stable pole in this hectic world, thank
you for always staying calm around me and exploring the world of
robotics software. Most of the work in this thesis would not have been possible
without you. Being a very opinionated person myself, I could not be more
thankful for having met you, Giovanni. You slowly convinced me of the concept of
learning-from-demonstration and never hesitated to start a heated discussion on
the topic of robotic manipulation. The most thrilling and exciting moments in
this journey were because of you, \#platonics. Mariano, thank you for keeping a
calm mind when around Giovanni and me, both when trying to figure out rotations
and when deciding on where to have lunch. I also want to thank Bruno, for
hinting me towards the geometric approaches, when I was looking for a meaningful
research direction.


Maxi, thank you for teaming up in this journey and for all the
healthy distractions from the, often times, toxic academic environment.


