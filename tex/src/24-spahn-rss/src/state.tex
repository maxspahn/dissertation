\section{Related Work}

Robotic mobile manipulation stands as a dynamic and expansive field of research, spurred by diverse potential applications and further fueled by prestigious
international competitions, such as DARPA’s Robotics Challenge
\cite{darpa_challenge}, RoboCup@Home~\cite{robocup_home}, the Amazon Picking
Challenge \cite{corbato2018integrating}, and RoboCup@Work \cite{robocup_work}. These
competitions are tailored to address distinct challenges and performance
criteria. Numerous research projects have yielded a significant number of
various mobile manipulation platforms. For an exhaustive overview of wheeled
mobile manipulation systems and the associated challenges, readers can refer to
\cite{overview1,overview2}. 

A supermarket mobile manipulator showcase has been presented in \cite{toyota2023} with a special focus on metrics in real-world settings and the importance of quantitative field experiments. Similar long-term fetch and carry experiments, yet in different environments, were carried out by Domel et al.~\cite{domel2017toward} in a factory environment and by Stibinger et al.~\cite{vstibinger2021mobile} in an outdoor competition to pick up and place simulated
construction materials. Instead of relying on a  Model Predictive Control formulation, such as presented in ~\cite{minniti2021model} for motion planning and control, we deploy a more reactive trajectory generation method \cite{ratliff2023fabrics}, and a task planning and execution approach that is more adaptive in the presence of disturbances.



