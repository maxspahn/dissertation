\iffalse 
- decisions/assumptions for retail environments
  - layout known
  - rough location of products, geometry, mass known in the format of a database
  - aiming at picking around 0.6 of the products, leaving out fruits and
  vegetable, very small items, and very heavy items
  - suction cups for grasping as it is simple and affordable
  - compliant control because of human-shared environment

- main goals
  - embrace recovery from failures, they are inevitable in dynamic environments
  - programming must be ongoing and realized by non-experts
  - everything should be running on-board
  - whole body control if advantageous

- hardware
  - franka emika panda robotic arm
  - boxer clearpath base
  - custom build suction gripper with long shaft
  - realsense D435 camera mounted on end-effector
  - Intel NUC compute unit
  - Alienware laptop for perception pipeline

- perception
  - object detection based on yolo
  - object classification based on product details
  - products can be added on-the-fly with one image

- trajectory generation
  - fabrics
  - visual servoing for item picking
  - state-feedback during exuction
  - recording using constraints in compliant mode
  - low-level controller: Velocity-PID outputting joint torques

- decision making
  - based on active inference for decision making
  - tasks are encoded as required states and action deduced based on that

- side-features
  - voice-over for real-time self-explanations
\fi

%As we aim at evaluating recent advancement relevant for mobile manipulation for
%human-shared environment, we decided to center this work around order-picking in
%supermarkets during opening-hours. Based on this, we can derive certain
%assumptions and decisions that guided our decisions throughout this work.
%In the following, we enumerate assumptions and justify them in the
%context of retail-environments. Then we explain the used-methods and their
%adaption to this application.






