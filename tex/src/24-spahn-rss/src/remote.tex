\subsection{Digital Twin for Remote Monitoring and Control}
\label{sec:digital_twin}

\begin{figure*}[t]
  \begin{center}
    \includegraphics[width=0.85\linewidth]{remote_v5.pdf}
  \end{center}
  \caption{Overview of the remote monitoring and control system. a) Laptop-based remote interface for monitoring and control system; b) Visualization of the robot within the actual retail environment; c) Tablet interface for on-the-go monitoring and task programming}\label{fig:remote_overview}
\end{figure*}

Herein, we introduce a digital twin mechanism to support remote monitoring and control of a mobile manipulator in a retail setting, as shown in Fig.~\ref{fig:remote_overview}. By scanning the environment in three dimensions, we construct a virtual model that accurately represents the workspace. The robot, when operational in a supermarket, is connected to this digital twin through Wi-Fi or 4G, enabling operators to monitor its status and issue commands from any remote location. The addition of a tablet interface allows for flexible monitoring and control by on-site staff, who can easily adjust the robot's course or teach it new tasks as needed.



