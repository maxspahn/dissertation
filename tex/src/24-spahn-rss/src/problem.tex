\section{Problem Statement}

This demonstration is concerned with mobile manipulation in
retail environments. Although the realization of such an
application is deemed simple, the actual implementation in a
human-shared environment must be reliable, robust and
fault-tolerant. In this work, we want to give a user's
perspective on what must be taken into consideration when
creating actual real-world application with mobile
manipulation. Doing so, we present key takeaways from this
project and emphasize on three main questions:

\begin{itemize}
  \item Assuming that errors may occur, how can we ensure fault-tolerant task
  assignment and task-planning?
  \item How can trajectory generation be robust, safe and easy to adapt by
  non-experts?
\end{itemize}

\subsection{Assumptions in the context of supermarkets}

Modern retails and supermarkets have access to a detailed information of all
products, this includes mass, geometry and location in the store. In this paper,
we assume that the robot can access this database to inform its decision.
Supermarkets are characterized by a large range of products, around 100.000
different products per store.  This work deliberately excludes
 products that require specialized grasping strategies or even specific hardware design in favor of reliability for the remaining products.
Moreover, in the interest of bridging the gap between robotics research and
applications on the market, achieving high reliability for a part of all
products is assumed to be favorable. Besides, we assume that products to
pick are visible from the front of the shelf. We explicitly focus on
in-store-picking during opening-hours, because warehouse automation is already
being tackled by redesigning the interior making collaborative robots obsolete
in such environments. Based on this focus, we generally favor safety over
execution speed. 
%\subsection{Main Goals}
%The overarching objectives guiding the development of the robotic system include
%the ability to embrace the inevitable recovery from failures in dynamic
%environments. We aim to enable ongoing programming by individuals without
%specialized robotics expertise. All processes are designed for on-board
%execution, minimizing reliance on external systems. Whole-body control
%strategies are employed when deemed advantageous for specific tasks.
