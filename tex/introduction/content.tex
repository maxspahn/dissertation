\chapter{Introduction} % (fold)
\label{cha:Introduction}

%	- ~1 context with issues arising
%	- (~ 0.5 common approaches)
%	- (~ 1 specifics to the case studies)
%	- ~1.5 common approaches and drawbacks (high-level)
%	- ~1 page new method intro high-level
%	- ~0.5 RQ/contributions
%	- ~0.5 overview

\section{Structure}

Problem in society
\begin{enumerate}
  \item Aging modern societies, changing demographics
  \item Shortage of cheap labour or nobody willing to do physically challenging
  task
  \item while advancements in natural language processing and computer vision
  led to large improvements, emobodied systems are still dull and simplistic
  \item the raise of collaboraitev robots promises safe robots that are
  inherently safe in the proximity of humans
  \item the concept of mobile manipulation, whether legged or on wheels, gives
  robotic hardware the same capabilities as humans, sparking the hope that
  labour shortage can be compensated with such systems
\end{enumerate}
Robotics as solution
\begin{enumerate}
  \item Robots to take over dull work, such as operating warehouses, restocking
  shelf, package delivery, food harvesting, etc
  \item Robots can work endless hours
  \item Robots are yet to be deployed in human-shared environments because of
  dynamics of such environments and safety constraints
  \item One important part for deploying robots in human shared environments is
  trajectory generation, what commands to send to the motors such that the robot
  is getting closer to its goal
  \item concluding, the hardware is capable of being deployed, but the methods
  for generating safe trajectories is yet to be found
\end{enumerate}
Common approaches and their drawbacks
\begin{enumerate}
  \item behind fences: linear motions or once-computed collision free global paths
  \item this approach works independent of duration for planning as it is done
  once
  \item such approaches fail to cope with changing environments
  \item lead to the raise of local trajectory generation methods
  \item pure control approaches, such as operational space control or impedence
  control, are fast and explainable but fail to integrate the full problem,
  including self-collision avoidance, collision-avoidance
  \item method centered around optimization problem, consisting of an objective
  function and constraints is favorable for safety guarantees and optimizality
  \item optimization-based methods have shown great sucess in autonomous driving
  but fail to meet real-time constraints for systems with more degrees of
  freedome
  \item learning based approaches are popular in the age of deep neural networks
  but fail to meet often fail to meet safety constraints and tend to overfit
  to the trained use-case, lack generalizability
\end{enumerate}
Geometric encoding as intermediate between control and trajectory generation
\begin{enumerate}
  \item trajectory generation must be reactive and generalizable
  \item cartesian impedance control offers a very fast way, but does not include
  many important components
  \item in cartesian impedence control, we model the problem as a spring-damper
  system
  \item that system is effectively a differential equation of second order for
  goal attraction
  \item the same could be done for any other components of trajectory generation
  \item set of ode's representing all the components
  \item the combinations describes a space in which some regions are desirable
  and some are not desirable
  \item the entire configuration space is thus equipped with a metric
  \item this shaped space is the encoding on which we can move towards a goal
  \item implicitevly solving for most components
  \item this concept has been formulaized as optimization fabrics
\end{enumerate}
Layout of this thesis
\begin{enumerate}
  \item introduction to optimization based techniques, specifically model
  predictive control
  \item introduction to differential geometry and optimization fabrics for
  trajectory generation
  \item model predictive control for whole-body control, this chapter highlights
  the issues with optimization-based methods
  \item dynamic optimization fabrics, this chapter makes important
  generalizations of fabrics to dynamic environments  
  \item symbolic optimization fabrics and automate tuning
  \item conclusion
\end{enumerate}

\section{Robots under the eyes of demographic change}

In a world grappling with the challenges of an aging population, shifting
demographics, and a scarcity of affordable labor for physically demanding tasks,
the need for technological solutions is more pronounced than ever. While
strides in natural language processing and computer vision have significantly
improved our office activities, embodied systems still
lag behind in sophistication.

Traditional robotic systems, though capable in certain tasks, often lack the
nuanced understanding and adaptability required for dynamic, human-shared
environments. The emergence of collaborative robots promises
safety in close proximity to humans. These robots are designed to interact
seamlessly and safely with their human counterparts, fostering a new era of
cooperation between man and machine. 
% This might already be problems
%While the hardware often gives the
%impressions of being capable, the algorithms generating autonomy are still
%failing to meet expectations. To tackle the challenges by demographic changes
%in modern societies, methods aiming for autonomy are yet to be found.

\subsection{Mobile manipulation}

One key feature of modern robots for human-shared environments, is mobility. The
combination of highly mobile ground vehicles and manipulators is referred to as
mobile manipulation. This concept enables robots to navigate and interact with
their surroundings in ways previously deemed challenging. By endowing robots
with the ability to move and manipulate objects in diverse environments, we
further open avenues for addressing the mentioned societal challenges.
Specifically, mobile manipulators could be used for tasks like warehouse
operations, restocking shelves, and package delivery to intricate processes like
food harvesting, robots offer a solution to alleviate humans from monotonous and
physically demanding work. Their tireless work ethic addresses challenges
related to labor shortages and reluctance to undertake arduous tasks.

\subsection{Challenges in robotics}

However, despite their potential, robots are noticeably absent from human-shared
environments. The complexities of such spaces, coupled with safety constraints,
present formidable challenges. This is where trajectory generation becomes
pivotal—a fundamental aspect in deploying robots in close proximity to humans.
It determines the commands sent to the motors, orchestrating a harmonious
movement that brings the robot closer to its goal while ensuring the safety of
its human counterparts.

Essentially, while the hardware is ready for deployment, the missing link lies
in methods for generating trajectories that prioritize safety. This dissertation
embarks on a journey to unravel this critical aspect of robotics, exploring
innovative trajectory generation methods that not only unlock the potential of
robotic hardware but also usher in an era where robots seamlessly coexist with
humans in shared spaces.

\section{Current Trajectory Generation Landscape: Unpacking Approaches and Their
Limitations}

In the quest for effective trajectory generation, various methods have emerged,
each with its set of advantages and drawbacks. A prevalent approach involves
confining robots behind fences, employing linear motions or precomputed
collision-free global paths. While such methods are independent of planning
duration, being computed once, their Achilles' heel lies in their incapacity to
adapt to dynamic, changing environments.

Recognizing this limitation has spurred the rise of local trajectory generation
methods. However, even within this domain, challenges persist. Pure control
approaches, exemplified by operational space control or impedance control, offer
speed and interpretability but fall short in incorporating the entirety of the
problem. They struggle with aspects like self-collision avoidance and fail to
provide a comprehensive solution for collision avoidance in complex
environments.

Another avenue explored centers around optimization problems, boasting an
objective function and constraints that favor safety guarantees and optimality.
While optimization-based methods have demonstrated success in autonomous driving
applications, they stumble when confronted with real-time constraints in systems
with increased degrees of freedom.

In the age of deep neural networks, learning-based approaches have gained
popularity. However, their drawback lies in their tendency to falter concerning
safety constraints, often succumbing to overfitting specific use cases during
training and lacking generalizability when confronted with new scenarios.

This dissertation navigates through these existing trajectories, meticulously examining their strengths and limitations. By critically analyzing the shortcomings of current approaches, we pave the way for innovative solutions in trajectory generation that address the ever-evolving challenges of dynamic and shared environments.

In this dissertation, we delve into the heart of mobile manipulation, with a
particular focus on trajectory generation — a pivotal aspect that dictates the
path these robots follow. As we explore the intricate dance between technology
and society, we aim to uncover innovative solutions that not only bridge
existing gaps but also pave the way for a future where robotic systems
seamlessly integrate into our daily lives.





