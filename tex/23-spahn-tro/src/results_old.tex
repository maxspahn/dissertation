In this part, experimental results on dynamic are presented and compared to findings in
\cite{Ratliff2020}. This empirical evidence confirming the theoritical findings highlights
the potential of the proposed dynamic fabrics. As the main advantage of dynamic fabrics
lays in the ability to integrate global plans in taskspace, we will use model predictive
control as a baseline. MPC offers trajectory following in the task space and is therefore
suited for comparasion. Besides, comparisions to static fabrics, as presented in
\cite{Ratliff2020} will be done. If not noted differently, the following motion planning
methods are compared:
\begin{compactitem}
  \item (static) fabrics as proposed in \cite{Ratliff2020}
  \item Model predictive control with $N=30, \Delta t=0.2$
  \item dynamic fabrics with time variant potentials
\end{compactitem}

\section{Point robots}%
\label{sec:point_robots}

The first experimental setup constists of a point mass operating in
the 2D-plane. Environments with different number of obstacles will be evaluated using the
different motion planning methods.

\subsection{General setup}%
\label{sub:general_setup}

Throughout this entire section some parameters will remain unchanged. A set of 10
particles (point robots) is given the same target and environment. All entities in this
set differ only by the initial velocity direction. The initial configuration is set to
$\q_0 = [3, 2]^T$ and $\norm{\qdot_0} = 1.0$. The velocity vectors are distributed
uniformly over the interval $[-\pi, \pi]$. The different initial velocities are
represented by different colors in plots. The goal position is set to $\q_{g} = [-2,
-1]^T$ for all of the following expirements. A fourth-order Runge-Kutta forward integration
schemes was used. The step size was fixed to $\dt = 0.01$ and a total time of $T=15s$ is
simulated. The boundaries in the joint space are respected using one-dimensional distance
task spaces as described in \cite{Ratliff2020}.

\subsection{Local minima}%
\label{sub:local_minima}

We introduce two different obstacle configurations for this robot.
\begin{compactitem}
  \item one obstacle at $\q_{o,1} = [0.5, 0.0]^T$ with radius $r=0.8$
  \item two obstacles at $\q_{o,1} = [0.5, 0.0]^T$ and
          $\q_{o,2} = [0.0, 1.0]^T$, both with radius $r=0.8$
\end{compactitem}
The resulting trajectories are plotted in \cref{fig:pointRobot_static}.
\begin{figure}
  \begin{subfigure}[]{0.5\textwidth}
  \begin{center}
    \includegraphics[width=1.0\linewidth]{pointRobot/static/output_trajectories}
  \end{center}
  \caption{trajectories}
  \label{fig:pointRobot_static_trajectories}
  \end{subfigure}%
  \begin{subfigure}[]{0.5\textwidth}
  \begin{center}
    \includegraphics[width=1.0\linewidth]{pointRobot/static/output_solvertimes}
  \end{center}
  \caption{solvertimes}
  \label{fig:pointRobot_static_solvertimes}
  \end{subfigure}
  \caption{Two different scenaries with one and two obstacles respectively. The generated
trajectories are on the right. Different colors represent different initial moving
directions. Using fabrics as proposed in \cite{Ratliff2020} in the top row and model
predicitive control in the bottom row. The distribution of the solving times are
visualized on the right side.}%
  \label{fig:pointRobot_static}
\end{figure}

\subsection{Dynamic fabrics using basic splines}

To overcome the problem of local minima, we make use of simple basic splines. For the
above described problem, three control points are sufficient.
This spline is applied to MPC and fabrics to guide the robots. The resulting trajectories
and solver times are visualized in \cref{fig:pointRobot_dynamic}.
\begin{figure}
  \begin{subfigure}[]{0.5\textwidth}
  \begin{center}
    \includegraphics[width=1.0\linewidth]{pointRobot/dynamic/output_trajectories}
  \end{center}
  \caption{trajectories}
  \label{fig:pointRobot_dynamic_trajectories}
  \end{subfigure}%
  \begin{subfigure}[]{0.5\textwidth}
  \begin{center}
    \includegraphics[width=1.0\linewidth]{pointRobot/dynamic/output_solvertimes}
  \end{center}
  \caption{solvertimes}
  \label{fig:pointRobot_dynamic_solvertimes}
  \end{subfigure}
  \caption{Given a goal position and a corresponding basic spline, dynamic fabrics and MPC
succeed at avoiding the obstacle. Solvertimes are visualized on the right side.}%
  \label{fig:pointRobot_dynamic}
\end{figure}

\subsection{Obstructed spline}%
\label{sub:obstructed_spline}

While splines introduce global knowledge, it must be ensured that generated motions are
remain safe when the global plan is infeasible. An example is visualized in
\cref{fig:pointRobot_dynamic_obstruction}.
\begin{figure}
  \begin{subfigure}[]{0.5\textwidth}
  \begin{center}
    \includegraphics[width=1.0\linewidth]{pointRobot/dynamic_obstruction/output_trajectories}
  \end{center}
  \caption{trajectories}
  \label{fig:pointRobot_dynamic_obstruction_trajectories}
  \end{subfigure}%
  \begin{subfigure}[]{0.5\textwidth}
  \begin{center}
    \includegraphics[width=1.0\linewidth]{pointRobot/dynamic_obstruction/output_solvertimes}
  \end{center}
  \caption{solvertimes}
  \label{fig:pointRobot_dynami_obstructionc_solvertimes}
  \end{subfigure}
  \caption{Given a goal position and a corresponding basic spline (thick green), dynamic fabrics and MPC
succeed at avoiding the obstacle. Solvertimes are visualized on the right side.}%
  \label{fig:pointRobot_dynamic_obstruction}
\end{figure}





\subsection{Discussion}

In the presence of one convex obstacle (or a set of seperated convex obstacles), fabrics
with static attractors show good behavior conforming the findings in \cite{Ratliff2020}.
However, if two obstacles form a non-convex shape, the point is attracted to the potential
directly and thus trapped in the local minimum. Note, that collision avoidance is stil
achieved at all time. Given the same task, an MPC is able to solve both motion planning
problems with any difficulty. The time horizon allows to circumnavigate both obstacles
smoothly at the cost of having slightly higher solving times
\cref{fig:pointRobot_static_solvertimes}.

When a global plan is integrated in the form of a basic spline, the problem of local
minima is solved.
This should not be suprising as
global planning introduce global, prior knowledge to the motion planning. However,
fabrics, in the original work, have not been equipped with the ability to integrate global
planning using time parametrized global paths. Equally, the MPC solves the same guided
motion planning successfully with comparable solver times, see
\cref{fig:pointRobot_dynamic}.
Both methods remain safe when the global paths is obstructed by a third obstacles with
similar solving times, see \cref{fig:pointRobot_dynamic_obstruction}

\section{Planar robots: static arms}
Next, various planar kinematic chains will be investigated in combination with the three
presented methods (static fabrics, MPC, dynamic fabrics). Initially, the task space is
defined as the position of the end-effector, i.e. the 2D-plane. In the presented examples,
the goal is defined as $\x_{g} = [2, 3]^T$. The following kinematic chains are compared:
\begin{compactitem}
  \item four-link arm with $l_i = 1.0$
  \item five-link arm with $l_i = 1.0$
\end{compactitem}

Collision avoidance is defined for the multiple points on the robot to ensure safety.
Specifically, for the MPC, inequality constraints for the end point of each link as well as their
center points are formulated for every obstacles. Similarly, obstacle avoiding geometries
are specified for the same points and all obstacles for the fabrics.

\begin{figure}
  \begin{subfigure}[]{0.5\textwidth}
  \begin{center}
    \includegraphics[height=1.0\linewidth]{planarRobot/n4/static/output}
  \end{center}
  \caption{Trajectories}
  \label{fig:planarRobot_n4_static_trajectories}
  \end{subfigure}%
  \begin{subfigure}[]{0.5\textwidth}
  \begin{center}
    \includegraphics[height=1.0\linewidth]{planarRobot/n4/static/output_solvertimes}
  \end{center}
  \caption{Solvertimes}
  \label{fig:planarRobot_n4_static_solvertimes}
  \end{subfigure}
  \caption{Results for planar robot with four links ($l_i = 1$). On the right,
end-effector trajectories are shown in red. The initial configuration is visualized in
green, an intermediate configuration in light grey and the finial configuration in black.
The green circle is the end-effector goal. Boxplots on the right show solving times for
both methods.}%
  \label{fig:planarRobot_n4_static}
\end{figure}

\iffalse
\begin{figure}
  \begin{subfigure}[]{0.5\textwidth}
  \begin{center}
    \includegraphics[height=1.0\linewidth]{planarRobot/n4/dynamic/output}
  \end{center}
  \caption{Trajectories}
  \label{fig:planarRobot_n5_dynamic_trajectories}
  \end{subfigure}%
  \begin{subfigure}[]{0.5\textwidth}
  \begin{center}
    \includegraphics[height=1.0\linewidth]{planarRobot/n4/dynamic/output_solvertimes}
  \end{center}
  \caption{Solvertimes}
  \label{fig:planarRobot_n4_dynamic_solvertimes}
  \end{subfigure}
  \caption{Results for planar robot with four links using basic splines as global
guidance. Both methods allow following the global path. Solvertimes (right) highlight the
difference between fabric and MPC.}%
  \label{fig:planarRobot_n4_dynamic}
\end{figure}
\fi

\begin{figure}
  \begin{subfigure}[]{0.5\textwidth}
  \begin{center}
    \includegraphics[height=1.0\linewidth]{planarRobot/n5/static/output}
  \end{center}
  \caption{trajectories}
  \label{fig:planarRobot_n5_static_trajectories}
  \end{subfigure}%
  \begin{subfigure}[]{0.5\textwidth}
  \begin{center}
    \includegraphics[height=1.0\linewidth]{planarRobot/n5/static/output_solvertimes}
  \end{center}
  \caption{solvertimes}
  \label{fig:planarRobot_n5_static_solvertimes}
  \end{subfigure}
  \caption{Results for planar robot with five links ($l_i=1.0$) for one ore two obstacles.
Trajectories for static fabrics (top row) and MPC (bottom row) are attracted to the local
minumum and become stationary when two obstacles are present. Solvertimes for static
fabrics are about a sixth of those of MPC (right plot).}%
  \label{fig:planarRobot_n5_static}
\end{figure}

\begin{figure}
  \begin{subfigure}[]{0.5\textwidth}
  \begin{center}
    \includegraphics[height=1.0\linewidth]{planarRobot/n5/dynamic/output}
  \end{center}
  \caption{trajectories}
  \label{fig:planarRobot_n5_dynamic_trajectories}
  \end{subfigure}%
  \begin{subfigure}[]{0.5\textwidth}
  \begin{center}
    \includegraphics[height=1.0\linewidth]{planarRobot/n5/dynamic/output_solvertimes}
  \end{center}
  \caption{solvertimes}
  \label{fig:planarRobot_n5_dynamic_solvertimes}
  \end{subfigure}
  \caption{Results for planar robot with five links given end-effector guidance (green
curve). Using the prior, global knowledge fabric and MPC circumnavigate both obstacles
with different solving times (right).}%
  \label{fig:planarRobot_n5_dynamic}
\end{figure}

\subsection{Discussion}
Fabrics and MPC schemes allow to directly formulate motion planning goals in the
workspace while also integrating collision avoidance for arbitrary links on the kinematic
chain, see \cref{fig:planarRobot_n4_static_trajectories}. Opposito to point masses, solver
times for MPC are about three times higher for $n=4$ compared to fabrics
(\cref{fig:planarRobot_n4_static_solvertimes}).

Both methods, as local planning methods, are prone to get trapped in local minima, as
observed for $n=5$ in \cref{fig:planarRobot_n5_static_trajectories}.
Levereging dynamic fabrics, fabrics can overcome this shortcoming, similarly to trajectory
following using MPC \cref{fig:planarRobot_n5_dynamic_trajectories}. However, dynamic
fabrics show significantly lower solver times compared to MPC in this case
(\cref{fig:planarRobot_n5_dynamic_solvertimes}).

\section{Franka Emika Panda}
In this section, results for a real robot are presented. The environment consists of a
Franka Emika Panda robot and five obstacles. The compared methods are, as before, static
fabrics, MPC and dynamic fabrics. In the first experiment, the obstacles are placed in a
way that both MPC and static succeed without guiding. For the second experiment, the scene
is more occluded by the obstacles. This environment is kept while introducing spline
guiding in the third experiment for the MPC and dynamic fabrics. Obstacle avoidance is
realized using distance functions between the obstacles and all links on the kinematic
chain for all methods. Results are quantified using the position plots of the
end-effector, minimal distance to the obstacles and solver times.
The results are visualized in
\cref{fig:pandaRobot_static,fig:pandaRobot_static_infeasible,fig:pandaRobot_dynamic}.

\begin{figure}
  \begin{subfigure}[]{0.3\textwidth}
  \begin{center}
    \includegraphics[height=2.0\linewidth]{pandaRobot/static/output_ee}
  \end{center}
  \caption{Trajectories}
  \label{fig:pandaRobot_static_trajectories}
  \end{subfigure}%
  \begin{subfigure}[]{0.3\textwidth}
  \begin{center}
    \includegraphics[height=2.0\linewidth]{pandaRobot/static/output_distances}
  \end{center}
  \caption{Clearance}
  \label{fig:pandaRobot_static_distances}
  \end{subfigure}%
  \begin{subfigure}[]{0.3\textwidth}
  \begin{center}
    \includegraphics[height=2.0\linewidth]{pandaRobot/static/output_solvertimes}
  \end{center}
  \caption{Solvertimes}
  \label{fig:pandaRobot_static_solvertimes}
  \end{subfigure}
  \caption{Results for panda robot. From left to right: End-effector position and goal in
dashed lines(left). Clearance computed as minimum distance between any link and any
obstacle in the environment (center). Solving times (right).
}%
  \label{fig:pandaRobot_static}
\end{figure}

\begin{figure}
  \begin{subfigure}[]{0.3\textwidth}
  \begin{center}
    \includegraphics[height=2.0\linewidth]{pandaRobot/static_infeasible/output_ee}
  \end{center}
  \caption{Trajectories}
  \label{fig:pandaRobot_static_infeasible_trajectories}
  \end{subfigure}%
  \begin{subfigure}[]{0.3\textwidth}
  \begin{center}
    \includegraphics[height=2.0\linewidth]{pandaRobot/static_infeasible/output_distances}
  \end{center}
  \caption{Clearance}
  \label{fig:pandaRobot_static_infeasible_distances}
  \end{subfigure}%
  \begin{subfigure}[]{0.3\textwidth}
  \begin{center}
    \includegraphics[height=2.0\linewidth]{pandaRobot/static_infeasible/output_solvertimes}
  \end{center}
  \caption{Solvertimes}
  \label{fig:pandaRobot_static_infeasible_solvertimes}
  \end{subfigure}
  \caption{Results for panda robot. From left to right: End-effector position and goal in
dashed lines(left). Clearance computed as minimum distance between any link and any
obstacle in the environment (center). Solving times (right).
}%
  \label{fig:pandaRobot_static_infeasible}
\end{figure}

\begin{figure}
  \begin{subfigure}[]{0.3\textwidth}
  \begin{center}
    \includegraphics[height=2.0\linewidth]{pandaRobot/dynamic/output_ee}
  \end{center}
  \caption{Trajectories}
  \label{fig:pandaRobot_dynamic_trajectories}
  \end{subfigure}%
  \begin{subfigure}[]{0.3\textwidth}
  \begin{center}
    \includegraphics[height=2.0\linewidth]{pandaRobot/dynamic/output_distances}
  \end{center}
  \caption{Clearance}
  \label{fig:pandaRobot_dynamic_distances}
  \end{subfigure}%
  \begin{subfigure}[]{0.3\textwidth}
  \begin{center}
    \includegraphics[height=2.0\linewidth]{pandaRobot/dynamic/output_solvertimes}
  \end{center}
  \caption{Solvertimes}
  \label{fig:pandaRobot_dynamic_solvertimes}
  \end{subfigure}
  \caption{Results for panda robot with dynamic fabrics and equally guided MPC.
From left to right: End-effector position and goal in
dashed lines(left). Clearance computed as minimum distance between any link and any
obstacle in the environment (center). Solving times (right).
}%
  \label{fig:pandaRobot_dynamic}
\end{figure}

