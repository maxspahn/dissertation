\section{Conclusion}%
\label{sec:conclusion}

In this paper, we have generalized optimization fabrics to dynamic
environments. We have proven that our proposed \acl{df} are convergent to
reference paths and can thus compute motion for path following tasks
(\cref{lem:dynamically_energized_fabrics}). Besides, we have proposed an
extension to optimization fabrics (and thus also \ac{df}) for
non-holonomic robots. This allows the application of this framework to a wider
range of robotic applications and ultimately allows the deployment to many
mobile manipulators in dynamic environments.

These theoretical findings were confirmed in various experiments. First, the
quantitative comparisons showed that \acl{sf} outperforms \ac{mpc} in terms of
solver time while maintaining similar performance in terms of goal-reaching and
success rate. 
The improved performance with optimization fabrics might be caused
by the different metric for goal
reaching compared to \acl{mpc}. An integration of
non-Riemannian metrics into an MPC formulation should be further
investigated in the future.

Verifying our theoretical derivations for \ac{df}, the experiments
showed that the deviation error for path following tasks is decreased compared
to \ac{sf}. Similarly, environments with moving obstacles and humans showed
increased clearance while maintaining low computational costs and execution
times. Thus, \ac{df} overcome an important drawback of
\ac{sf}~\cite{Ratliff2020,Wyk2022}, where collision avoidance with moving
obstacle is solved purely by the high frequency at which optimization fabrics
can be computed. Moreover, the generalization did not increase the solving time
compared to \ac{sf}. Unlike the original work on optimization fabrics, this
generalization allows the deployment to dynamic environments where velocity
estimates of moving obstacles are available.

Direct sensor integration in optimization fabrics might be feasible in future
works to overcome the shortcomings of perception pipelines for collision
avoidance. For the trajectory path tasks in this paper, we
used a simple global path generated in workspace. As \ac{df} integrate
global path in arbitrary manifolds, improving the global planning phase could
be further investigated. We expect this to be beneficial when robotics tasks
are constantly changing and task planning is required.
