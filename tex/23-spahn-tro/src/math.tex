\section{Background}%
\label{sec:mathematics}

In the previous section, we have highlighted that optimization fabrics represent a
powerful tool for reactive motion generation. Since we generalize this concept, this
section aims at familiarizing the reader with key findings on optimization fabrics and
recalling some of the basic notations known from differential geometry. 
We first give an overview on how optimization fabrics are used for motion generation
and how the components are derived and composed. Then the theoretical foundations
are summarized from \cite{Cheng2020,Ratliff2020}.

\subsection{Motion generation using optimization fabrics}%
\label{sub:trajectory_generation_using_optimization_fabrics}

When using optimization fabrics for motion generation, all components including constraints and 
goal attraction are designed as second order differential equations. If specific design
rules for these equations are respected, all components can be combined to 
form a converging motion generator. Specifically, the following steps are performed:
%
\begin{enumerate}
  \item Design path-consistent geometries in suited manifolds of the configuration space
    (\cref{eq:homogeneous}).
  \item Design corresponding Finsler energies defining the importance metric in this manifold
    (\cref{sub:conservative_fabrics_and_energization}).
  \item Energize all geometries with the associated Finsler energies 
    (\cref{sub:conservative_fabrics_and_energization}).
  \item Pull back the energized systems into the configuration space and sum them 
    (\cref{sub:operations_on_specs}).
  \item Force the combined system with a differentiable potential. As a composition of optimization fabrics, 
    the resulting trajectory converges towards the potential's minimum
    (\cref{sub:optimization_fabrics}).
\end{enumerate}
In the following, we introduce the reader to the theory of optimization fabrics
and recall important findings from \cite{Ratliff2020}.

\subsection{Configurations and task variables}%
\label{sub:configurations_and_task_variables}

We denote $\q\in\Q\subset\Rn$ a
configuration of the robot with $n$ its degrees of freedom; \Q{} is the configuration space of the generalized coordinates
of the system. Generally, $\q(t)$ defines the robot's configuration at time $t$, so that 
\qdot, \qddot{} define the instantaneous derivatives of the robot's configuration.
Similarly, we assume
that there is a set of task variables $\xj\in\Xj\subset\Rmj$ with variable dimension
$m_j \leq n$. The task manifold \Xj{} defines an arbitrary manifold of the configuration
space \Q{} in which a robotic task can be represented. 
Further, we assume that there is a differential map
$\map_j:\Rn\rightarrow\Rmj$ that relates the configuration space to the $j^{th}$ task
space. For example, when a task variable is defined as the end-effector position, then
$\map_j$ is the positional part of the forward kinematics. On the other hand, if a task
variable is defined to be the joint position, then $\map_j$ is the identity function. 
In the following, we drop the subscript $j$ in most cases for readability when the
context is clear.

In this work, we assume that \map{} is smooth and twice differentiable so that the Jacobian is
defined as
\begin{equation}
  \J = \derf{\q}{\map} \in \mathcal{R}^{m\times n}, 
\end{equation}
or $\J = \der{\q}{\map}$ for short.
Thus, we can write the total time derivatives of \x{} as
\begin{align}
  \xdot &= \J\qdot \\
  \xddot &= \J\qddot + \Jdot\qdot.
\end{align}

\subsection{Spectral semi-sprays}%
\label{sub:spectral_semi_sprays}

Inspired by simple mechanics (e.g., the simple pendulum), the framework of optimization
fabrics designs motion generation as second-order dynamical
systems $\xddot = \pi(\x,\xdot)$~\cite{Cheng2020,Ratliff2020}. While higher-order systems seem feasible, their
implementation on robots is much more challenging, as higher order configuration space
derivatives would be required. The trajectory generator is defined by the differential equation
$\M\xddot + \f = 0$, where $\M(\x,\xdot)$ and $\f(\x,\xdot)$ are functions of position and
velocity. Besides, \M{} is symmetric and invertible. Such systems $\Spec = \spec$ are
known as \textit{spectral semi-sprays}, or \textit{specs} for short.  When the space of
the task variable is clear from the context, we drop the subscript.  Then, the trajectory
is computed as the solution to the system $\xddot=-\M^{-1}\f$.

\subsection{Operations on specs}%
\label{sub:operations_on_specs}
Next, the two fundamental operations for specs, transformation between spaces and
summation, are introduced.

Given a differential map $\map: \Q\rightarrow\X$ and a spec \spec{}, the \textit{pullback}
is defined as 
\begin{equation}
  \pull{\map}{\spec} = {\left(\Jt\M\J, \Jt(\f+\M\Jdot\qdot)\right)}_{Q}.
\end{equation}
The pullback allows converting between two distinct manifolds (e.g. a spec could be 
defined in the robot's workspace and being pulled into the robot's configuration space using
the pullback with \map{} being the forward kinematics).

For two specs, $\Spec_1 = {\left(\M_1,\f_1\right)}_{\X}$ and 
$\Spec_2 = {\left(\M_2,\f_2\right)}_{\X}$, their \textit{summation} is defined by:
\begin{equation}
  \Spec_1 + \Spec_2 = {\left(\M_1 + \M_2, \f_1 + \f_2\right)}_{\X}.
\end{equation}

\subsection{Optimization fabrics}%
\label{sub:optimization_fabrics}

Optimization fabrics form a special class of specs, and thus they inherit their properties,
specifically the previously defined operations of \textit{summation} and \textit{pullback}.
First, let us introduce a finite and differentiable potential
function $\forc(\x)$ defined in a task manifold \X{}. 
Then, the modified spec $\Spec_{\forc} = \left(\M,\f + \der{\x}{\forc}\right)$
is called the \textit{forced variant} of $\Spec = \spec$.
Only if the trajectory $\x(t)$ generated by the forced spec converges to the minimum of \forc{}, 
the spec is said to form an \textit{optimization fabric}.
When the spec only converges to the minimum when equipped with a damping term,
$\left(\M,\f + \der{\x}{\forc} + \mat{B}\xdot\right)$, 
it forms a \textit{frictionless fabric}~\cite[Definition 4.4]{Ratliff2020}. 
Note that the mechanical system of a pendulum forms a frictionless fabric, as it optimizes
the potential function defined by gravity when being damped (i.e., it eventually comes to
rest at the configuration with minimal potential energy)
%, see Appendix
%\ref{app:sub:motion_as_second_order_systems} for a detailed discussion on the single
%pendulum).

In the following, methods to construct optimization fabrics, or \textit{fabrics} for
short, are summarized: the definitions of conservative fabrics and energization are
introduced.

\subsection{Conservative fabrics and energization}%
\label{sub:conservative_fabrics_and_energization}

While the previous subsection defined what criteria are required for a spec to form
an optimization fabric, the theory on conservative fabrics and energization 
offers a simple way of generating such special specs. As a full summary of the theory
on optimization fabrics and their construction is out of scope here, this
subsection only provides an outline of the theory and the reader is referred to 
\cite{Ratliff2021,Wyk2022} for detailed derivations.

In the context of fabrics, the term \textit{energy} describes a scalar quantity that
changes as the system evolves over time.  Although this quantity has a physical meaning in
natural systems (e.g., kinetic energy), it can be arbitrarily defined for motion generation.
Generally, specs and optimization fabrics do not conserve an energy, but when they do, we
call them \textit{conservative specs}.  A stationary Lagrangian~\cite[Definition
4.11]{Ratliff2020} is one definition for an energy for which the corresponding spec, known
as the Lagrangian spec $\Spec_{\le} = \left(\Me,\fe\right)$, is obtained by applying the
Euler-Lagrange equations. Importantly, Lagrangian specs conserve energy and do thus belong
to the class of \textit{conservative specs}.  It was proven that an unbiased
(\cite[Definition~4.11]{Ratliff2020}) Lagrangian spec forms a frictionless
fabric~\cite[Proposition~4.18]{Ratliff2020}. Such fabrics are analogously called
\textit{conservative fabrics}.  There are two classes of conservative fabrics: Lagrangian
fabrics (i.e., the defining energy is a Lagrangian) and the more specific subclass of
Finsler fabrics (i.e., the
defining energy is a Finsler structure~\cite[Definition 5.4]{Ratliff2020}). 

The operation of \textit{energization} transforms a given differential equation into
a conservative spec.
Specifically, given an unbiased energy Lagrangian \le{} with boundary conforming
\Me{}~\cite[Definition~4.6]{Ratliff2020} and
lower bounded energy \he{}, an unbiased spec of form $\Spec_{\vec{h}} = (\mat{I},\vec{h})$
is transformed into a frictionless fabric using energization as
\begin{equation}
  \begin{split}
  S_{\vec{h}}^{\le} &= \text{energize}_{\le}\{S_{\vec{h}}\} \\
    &= (\Me, \fe + \Pe[\Me\vec{h} - \fe]), 
  \end{split}
\end{equation}
where $\Pe = \Me\left(\Me^{-1} - \frac{\xdot\xdot^T}{\xdot^T\Me\xdot}\right)$ is an
orthogonal projector.
Energized specs maintain the energy of the Lagrangian and generally change
the trajectory of the underlying spec $\Spec_{\vec{h}}$.
However, if 
\begin{enumerate}
  \item $\Spec_{\vec{h}} = (\mat{I},\vec{h})$ is homogeneous of degree 2,
    \begin{equation}\vec{h}(\x, \alpha\xdot) = \alpha^2\vec{h}(\x, \xdot)\label{eq:homogeneous}\end{equation}
    and
  \item the energizing Lagrangian is a Finsler structure, 
\end{enumerate}
the resulting energized spec forms a frictionless fabric for which the trajectory matches
the original trajectory of $\Spec_{\vec{h}}$. We refer to energized fabrics with that
property as \textit{geometric fabrics}. Geometric fabrics form the building blocks for
motion generation with optimization fabrics.
Practically, energization equips the individual components of the planning problem
with a metric when being combined with other components.


\subsection{Experimental results fabrics}%
\label{sub:experimental_results_fabrics}

The theory explained above was tested on several simple kinematic chains
in~\cite{Ratliff2020,Ratliff2021}. As fabrics design motion as a summation of several
differential equations, each representing a specific constraint to the motion, it is
possible to sequentially design motion~\cite{Ratliff2020}. This procedure allows to
carefully tune individual components without harming the others. The application to a
planar arm in a goal-reaching setup was successfully tested in~\cite{Ratliff2020}. Here,
the authors illustrated how the resulting motion can be modified arbitrarily by the user
by adding additional constraints or preferences.

Although important concepts and findings on optimization fabrics were summarized in this
section, we refer to~\cite{Ratliff2020} for a more in-depth presentation of optimization
fabrics. In the following, we generalize the framework of optimization fabrics to dynamic settings.
