\documentclass[nativefonts]{TUD-dissertation2020}

%% Turn off page numbering for the propositions and make the margins on both
%% sides equal and symmetrical.
\geometry{twoside=false}
\pagestyle{empty}
\newcommand{\addmytitle}{\title[\textcolor{tudelft-cyan}{Coupled Local Motion
Planning for Mobile Manipulators}]{Geometric Control}}


\begin{document}

%% Specify the title and author of the thesis. This information will be used on
%% both the English and Dutch side and in the metadata of the final PDF.
\addmytitle
\author{Max}{Spahn}

\begin{center}

{\Large\titlefont\bfseries Propositions}

\bigskip

accompanying the dissertation

\bigskip

%% Print the title.
{\makeatletter
\titlestyle\bfseries\large\@title
\makeatother}

%% Print the optional subtitle.
{\makeatletter
\ifx\@subtitle\undefined\else
    \titlefont\titleshape\@subtitle
\fi
\makeatother}

\bigskip

by

\bigskip

%% Print the full name of the author.
\makeatletter
{\large\titlefont\bfseries\@firstname\ {\titleshape\@lastname}}
\makeatother

\end{center}

\bigskip
\bigskip

\begin{enumerate}

  \item Safety in the sense of collision avoidance is inherently in confilict
    with fast manipuation. (This thesis)
  \item In robotics, robustness is more valuable than optimality and formal
    guarantees. (This thesis)
  \item Optimization fabrics create organic motions. (This thesis)
  \item Engineering sciences should embrace failure.
  \item Perception is inaccurate, so the paradigm, sense-plan-act, is prone to
    fail.
  \item Academic publications in robotics are of little use for advancing the
    field without proper software.
  \item In software engineering, comments are a manifestation of failure.
  \item Major advancements in robotics are unlikely to come from academic
    instituations, but rather from companies.
  \item Effective peer-reviewing requires a countable reward system.

\end{enumerate}

\bigskip
\bigskip

%% Apart from the name and title of the supervisor, the following text is
%% dictated by the promotieregelement.
\begin{center}
These propositions are regarded as opposable and defendable, and have been approved as such by the promotor prof.\ dr.\ M.\ Wisse.
\end{center}

\clearpage
{\selectlanguage{dutch}

\begin{center}

{\Large\titlefont\bfseries Stellingen}

\bigskip

behorende bij het proefschrift

\bigskip

%% Print the title.
{\makeatletter
\titlestyle\bfseries\large\@title
\makeatother}

%% Print the optional subtitle.
{\makeatletter
\ifx\@subtitle\undefined\else
    \titlefont\titleshape\@subtitle
\fi
\makeatother}

\bigskip

door

\bigskip

%% Print the full name of the author.
\makeatletter
{\large\titlefont\bfseries\@firstname\ {\titleshape\@lastname}}
\makeatother

\end{center}

\bigskip
\bigskip

\begin{enumerate}

\item Stelling 1.
\item Stelling 2.
\item Stelling 3.
\item Stelling 4.
\item Stelling 5.
\item Stelling 6.
\item Stelling 7.
\item Stelling 8.
\item Stelling 9.
\item Stelling 10.

\end{enumerate}

\bigskip
\bigskip

%% Apart from the name and title of the supervisor, the following text is
%% dictated by the promotieregelement.
\begin{center}
Deze stellingen worden opponeerbaar en verdedigbaar geacht en zijn als zodanig goedgekeurd door de promotor prof.\ dr.\ M.\ Wisse.
\end{center}

}

\end{document}

